
% Default to the notebook output style

    


% Inherit from the specified cell style.




    
\documentclass[11pt]{article}

    
    
    \usepackage[T1]{fontenc}
    % Nicer default font (+ math font) than Computer Modern for most use cases
    \usepackage{mathpazo}

    % Basic figure setup, for now with no caption control since it's done
    % automatically by Pandoc (which extracts ![](path) syntax from Markdown).
    \usepackage{graphicx}
    % We will generate all images so they have a width \maxwidth. This means
    % that they will get their normal width if they fit onto the page, but
    % are scaled down if they would overflow the margins.
    \makeatletter
    \def\maxwidth{\ifdim\Gin@nat@width>\linewidth\linewidth
    \else\Gin@nat@width\fi}
    \makeatother
    \let\Oldincludegraphics\includegraphics
    % Set max figure width to be 80% of text width, for now hardcoded.
    \renewcommand{\includegraphics}[1]{\Oldincludegraphics[width=.8\maxwidth]{#1}}
    % Ensure that by default, figures have no caption (until we provide a
    % proper Figure object with a Caption API and a way to capture that
    % in the conversion process - todo).
    \usepackage{caption}
    \DeclareCaptionLabelFormat{nolabel}{}
    \captionsetup{labelformat=nolabel}

    \usepackage{adjustbox} % Used to constrain images to a maximum size 
    \usepackage{xcolor} % Allow colors to be defined
    \usepackage{enumerate} % Needed for markdown enumerations to work
    \usepackage{geometry} % Used to adjust the document margins
    \usepackage{amsmath} % Equations
    \usepackage{amssymb} % Equations
    \usepackage{textcomp} % defines textquotesingle
    % Hack from http://tex.stackexchange.com/a/47451/13684:
    \AtBeginDocument{%
        \def\PYZsq{\textquotesingle}% Upright quotes in Pygmentized code
    }
    \usepackage{upquote} % Upright quotes for verbatim code
    \usepackage{eurosym} % defines \euro
    \usepackage[mathletters]{ucs} % Extended unicode (utf-8) support
    \usepackage[utf8x]{inputenc} % Allow utf-8 characters in the tex document
    \usepackage{fancyvrb} % verbatim replacement that allows latex
    \usepackage{grffile} % extends the file name processing of package graphics 
                         % to support a larger range 
    % The hyperref package gives us a pdf with properly built
    % internal navigation ('pdf bookmarks' for the table of contents,
    % internal cross-reference links, web links for URLs, etc.)
    \usepackage{hyperref}
    \usepackage{longtable} % longtable support required by pandoc >1.10
    \usepackage{booktabs}  % table support for pandoc > 1.12.2
    \usepackage[inline]{enumitem} % IRkernel/repr support (it uses the enumerate* environment)
    \usepackage[normalem]{ulem} % ulem is needed to support strikethroughs (\sout)
                                % normalem makes italics be italics, not underlines
    

    
    
    % Colors for the hyperref package
    \definecolor{urlcolor}{rgb}{0,.145,.698}
    \definecolor{linkcolor}{rgb}{.71,0.21,0.01}
    \definecolor{citecolor}{rgb}{.12,.54,.11}

    % ANSI colors
    \definecolor{ansi-black}{HTML}{3E424D}
    \definecolor{ansi-black-intense}{HTML}{282C36}
    \definecolor{ansi-red}{HTML}{E75C58}
    \definecolor{ansi-red-intense}{HTML}{B22B31}
    \definecolor{ansi-green}{HTML}{00A250}
    \definecolor{ansi-green-intense}{HTML}{007427}
    \definecolor{ansi-yellow}{HTML}{DDB62B}
    \definecolor{ansi-yellow-intense}{HTML}{B27D12}
    \definecolor{ansi-blue}{HTML}{208FFB}
    \definecolor{ansi-blue-intense}{HTML}{0065CA}
    \definecolor{ansi-magenta}{HTML}{D160C4}
    \definecolor{ansi-magenta-intense}{HTML}{A03196}
    \definecolor{ansi-cyan}{HTML}{60C6C8}
    \definecolor{ansi-cyan-intense}{HTML}{258F8F}
    \definecolor{ansi-white}{HTML}{C5C1B4}
    \definecolor{ansi-white-intense}{HTML}{A1A6B2}

    % commands and environments needed by pandoc snippets
    % extracted from the output of `pandoc -s`
    \providecommand{\tightlist}{%
      \setlength{\itemsep}{0pt}\setlength{\parskip}{0pt}}
    \DefineVerbatimEnvironment{Highlighting}{Verbatim}{commandchars=\\\{\}}
    % Add ',fontsize=\small' for more characters per line
    \newenvironment{Shaded}{}{}
    \newcommand{\KeywordTok}[1]{\textcolor[rgb]{0.00,0.44,0.13}{\textbf{{#1}}}}
    \newcommand{\DataTypeTok}[1]{\textcolor[rgb]{0.56,0.13,0.00}{{#1}}}
    \newcommand{\DecValTok}[1]{\textcolor[rgb]{0.25,0.63,0.44}{{#1}}}
    \newcommand{\BaseNTok}[1]{\textcolor[rgb]{0.25,0.63,0.44}{{#1}}}
    \newcommand{\FloatTok}[1]{\textcolor[rgb]{0.25,0.63,0.44}{{#1}}}
    \newcommand{\CharTok}[1]{\textcolor[rgb]{0.25,0.44,0.63}{{#1}}}
    \newcommand{\StringTok}[1]{\textcolor[rgb]{0.25,0.44,0.63}{{#1}}}
    \newcommand{\CommentTok}[1]{\textcolor[rgb]{0.38,0.63,0.69}{\textit{{#1}}}}
    \newcommand{\OtherTok}[1]{\textcolor[rgb]{0.00,0.44,0.13}{{#1}}}
    \newcommand{\AlertTok}[1]{\textcolor[rgb]{1.00,0.00,0.00}{\textbf{{#1}}}}
    \newcommand{\FunctionTok}[1]{\textcolor[rgb]{0.02,0.16,0.49}{{#1}}}
    \newcommand{\RegionMarkerTok}[1]{{#1}}
    \newcommand{\ErrorTok}[1]{\textcolor[rgb]{1.00,0.00,0.00}{\textbf{{#1}}}}
    \newcommand{\NormalTok}[1]{{#1}}
    
    % Additional commands for more recent versions of Pandoc
    \newcommand{\ConstantTok}[1]{\textcolor[rgb]{0.53,0.00,0.00}{{#1}}}
    \newcommand{\SpecialCharTok}[1]{\textcolor[rgb]{0.25,0.44,0.63}{{#1}}}
    \newcommand{\VerbatimStringTok}[1]{\textcolor[rgb]{0.25,0.44,0.63}{{#1}}}
    \newcommand{\SpecialStringTok}[1]{\textcolor[rgb]{0.73,0.40,0.53}{{#1}}}
    \newcommand{\ImportTok}[1]{{#1}}
    \newcommand{\DocumentationTok}[1]{\textcolor[rgb]{0.73,0.13,0.13}{\textit{{#1}}}}
    \newcommand{\AnnotationTok}[1]{\textcolor[rgb]{0.38,0.63,0.69}{\textbf{\textit{{#1}}}}}
    \newcommand{\CommentVarTok}[1]{\textcolor[rgb]{0.38,0.63,0.69}{\textbf{\textit{{#1}}}}}
    \newcommand{\VariableTok}[1]{\textcolor[rgb]{0.10,0.09,0.49}{{#1}}}
    \newcommand{\ControlFlowTok}[1]{\textcolor[rgb]{0.00,0.44,0.13}{\textbf{{#1}}}}
    \newcommand{\OperatorTok}[1]{\textcolor[rgb]{0.40,0.40,0.40}{{#1}}}
    \newcommand{\BuiltInTok}[1]{{#1}}
    \newcommand{\ExtensionTok}[1]{{#1}}
    \newcommand{\PreprocessorTok}[1]{\textcolor[rgb]{0.74,0.48,0.00}{{#1}}}
    \newcommand{\AttributeTok}[1]{\textcolor[rgb]{0.49,0.56,0.16}{{#1}}}
    \newcommand{\InformationTok}[1]{\textcolor[rgb]{0.38,0.63,0.69}{\textbf{\textit{{#1}}}}}
    \newcommand{\WarningTok}[1]{\textcolor[rgb]{0.38,0.63,0.69}{\textbf{\textit{{#1}}}}}
    
    
    % Define a nice break command that doesn't care if a line doesn't already
    % exist.
    \def\br{\hspace*{\fill} \\* }
    % Math Jax compatability definitions
    \def\gt{>}
    \def\lt{<}
    % Document parameters
    \title{Exploratory Data Analysis}
    
    
    

    % Pygments definitions
    
\makeatletter
\def\PY@reset{\let\PY@it=\relax \let\PY@bf=\relax%
    \let\PY@ul=\relax \let\PY@tc=\relax%
    \let\PY@bc=\relax \let\PY@ff=\relax}
\def\PY@tok#1{\csname PY@tok@#1\endcsname}
\def\PY@toks#1+{\ifx\relax#1\empty\else%
    \PY@tok{#1}\expandafter\PY@toks\fi}
\def\PY@do#1{\PY@bc{\PY@tc{\PY@ul{%
    \PY@it{\PY@bf{\PY@ff{#1}}}}}}}
\def\PY#1#2{\PY@reset\PY@toks#1+\relax+\PY@do{#2}}

\expandafter\def\csname PY@tok@w\endcsname{\def\PY@tc##1{\textcolor[rgb]{0.73,0.73,0.73}{##1}}}
\expandafter\def\csname PY@tok@c\endcsname{\let\PY@it=\textit\def\PY@tc##1{\textcolor[rgb]{0.25,0.50,0.50}{##1}}}
\expandafter\def\csname PY@tok@cp\endcsname{\def\PY@tc##1{\textcolor[rgb]{0.74,0.48,0.00}{##1}}}
\expandafter\def\csname PY@tok@k\endcsname{\let\PY@bf=\textbf\def\PY@tc##1{\textcolor[rgb]{0.00,0.50,0.00}{##1}}}
\expandafter\def\csname PY@tok@kp\endcsname{\def\PY@tc##1{\textcolor[rgb]{0.00,0.50,0.00}{##1}}}
\expandafter\def\csname PY@tok@kt\endcsname{\def\PY@tc##1{\textcolor[rgb]{0.69,0.00,0.25}{##1}}}
\expandafter\def\csname PY@tok@o\endcsname{\def\PY@tc##1{\textcolor[rgb]{0.40,0.40,0.40}{##1}}}
\expandafter\def\csname PY@tok@ow\endcsname{\let\PY@bf=\textbf\def\PY@tc##1{\textcolor[rgb]{0.67,0.13,1.00}{##1}}}
\expandafter\def\csname PY@tok@nb\endcsname{\def\PY@tc##1{\textcolor[rgb]{0.00,0.50,0.00}{##1}}}
\expandafter\def\csname PY@tok@nf\endcsname{\def\PY@tc##1{\textcolor[rgb]{0.00,0.00,1.00}{##1}}}
\expandafter\def\csname PY@tok@nc\endcsname{\let\PY@bf=\textbf\def\PY@tc##1{\textcolor[rgb]{0.00,0.00,1.00}{##1}}}
\expandafter\def\csname PY@tok@nn\endcsname{\let\PY@bf=\textbf\def\PY@tc##1{\textcolor[rgb]{0.00,0.00,1.00}{##1}}}
\expandafter\def\csname PY@tok@ne\endcsname{\let\PY@bf=\textbf\def\PY@tc##1{\textcolor[rgb]{0.82,0.25,0.23}{##1}}}
\expandafter\def\csname PY@tok@nv\endcsname{\def\PY@tc##1{\textcolor[rgb]{0.10,0.09,0.49}{##1}}}
\expandafter\def\csname PY@tok@no\endcsname{\def\PY@tc##1{\textcolor[rgb]{0.53,0.00,0.00}{##1}}}
\expandafter\def\csname PY@tok@nl\endcsname{\def\PY@tc##1{\textcolor[rgb]{0.63,0.63,0.00}{##1}}}
\expandafter\def\csname PY@tok@ni\endcsname{\let\PY@bf=\textbf\def\PY@tc##1{\textcolor[rgb]{0.60,0.60,0.60}{##1}}}
\expandafter\def\csname PY@tok@na\endcsname{\def\PY@tc##1{\textcolor[rgb]{0.49,0.56,0.16}{##1}}}
\expandafter\def\csname PY@tok@nt\endcsname{\let\PY@bf=\textbf\def\PY@tc##1{\textcolor[rgb]{0.00,0.50,0.00}{##1}}}
\expandafter\def\csname PY@tok@nd\endcsname{\def\PY@tc##1{\textcolor[rgb]{0.67,0.13,1.00}{##1}}}
\expandafter\def\csname PY@tok@s\endcsname{\def\PY@tc##1{\textcolor[rgb]{0.73,0.13,0.13}{##1}}}
\expandafter\def\csname PY@tok@sd\endcsname{\let\PY@it=\textit\def\PY@tc##1{\textcolor[rgb]{0.73,0.13,0.13}{##1}}}
\expandafter\def\csname PY@tok@si\endcsname{\let\PY@bf=\textbf\def\PY@tc##1{\textcolor[rgb]{0.73,0.40,0.53}{##1}}}
\expandafter\def\csname PY@tok@se\endcsname{\let\PY@bf=\textbf\def\PY@tc##1{\textcolor[rgb]{0.73,0.40,0.13}{##1}}}
\expandafter\def\csname PY@tok@sr\endcsname{\def\PY@tc##1{\textcolor[rgb]{0.73,0.40,0.53}{##1}}}
\expandafter\def\csname PY@tok@ss\endcsname{\def\PY@tc##1{\textcolor[rgb]{0.10,0.09,0.49}{##1}}}
\expandafter\def\csname PY@tok@sx\endcsname{\def\PY@tc##1{\textcolor[rgb]{0.00,0.50,0.00}{##1}}}
\expandafter\def\csname PY@tok@m\endcsname{\def\PY@tc##1{\textcolor[rgb]{0.40,0.40,0.40}{##1}}}
\expandafter\def\csname PY@tok@gh\endcsname{\let\PY@bf=\textbf\def\PY@tc##1{\textcolor[rgb]{0.00,0.00,0.50}{##1}}}
\expandafter\def\csname PY@tok@gu\endcsname{\let\PY@bf=\textbf\def\PY@tc##1{\textcolor[rgb]{0.50,0.00,0.50}{##1}}}
\expandafter\def\csname PY@tok@gd\endcsname{\def\PY@tc##1{\textcolor[rgb]{0.63,0.00,0.00}{##1}}}
\expandafter\def\csname PY@tok@gi\endcsname{\def\PY@tc##1{\textcolor[rgb]{0.00,0.63,0.00}{##1}}}
\expandafter\def\csname PY@tok@gr\endcsname{\def\PY@tc##1{\textcolor[rgb]{1.00,0.00,0.00}{##1}}}
\expandafter\def\csname PY@tok@ge\endcsname{\let\PY@it=\textit}
\expandafter\def\csname PY@tok@gs\endcsname{\let\PY@bf=\textbf}
\expandafter\def\csname PY@tok@gp\endcsname{\let\PY@bf=\textbf\def\PY@tc##1{\textcolor[rgb]{0.00,0.00,0.50}{##1}}}
\expandafter\def\csname PY@tok@go\endcsname{\def\PY@tc##1{\textcolor[rgb]{0.53,0.53,0.53}{##1}}}
\expandafter\def\csname PY@tok@gt\endcsname{\def\PY@tc##1{\textcolor[rgb]{0.00,0.27,0.87}{##1}}}
\expandafter\def\csname PY@tok@err\endcsname{\def\PY@bc##1{\setlength{\fboxsep}{0pt}\fcolorbox[rgb]{1.00,0.00,0.00}{1,1,1}{\strut ##1}}}
\expandafter\def\csname PY@tok@kc\endcsname{\let\PY@bf=\textbf\def\PY@tc##1{\textcolor[rgb]{0.00,0.50,0.00}{##1}}}
\expandafter\def\csname PY@tok@kd\endcsname{\let\PY@bf=\textbf\def\PY@tc##1{\textcolor[rgb]{0.00,0.50,0.00}{##1}}}
\expandafter\def\csname PY@tok@kn\endcsname{\let\PY@bf=\textbf\def\PY@tc##1{\textcolor[rgb]{0.00,0.50,0.00}{##1}}}
\expandafter\def\csname PY@tok@kr\endcsname{\let\PY@bf=\textbf\def\PY@tc##1{\textcolor[rgb]{0.00,0.50,0.00}{##1}}}
\expandafter\def\csname PY@tok@bp\endcsname{\def\PY@tc##1{\textcolor[rgb]{0.00,0.50,0.00}{##1}}}
\expandafter\def\csname PY@tok@fm\endcsname{\def\PY@tc##1{\textcolor[rgb]{0.00,0.00,1.00}{##1}}}
\expandafter\def\csname PY@tok@vc\endcsname{\def\PY@tc##1{\textcolor[rgb]{0.10,0.09,0.49}{##1}}}
\expandafter\def\csname PY@tok@vg\endcsname{\def\PY@tc##1{\textcolor[rgb]{0.10,0.09,0.49}{##1}}}
\expandafter\def\csname PY@tok@vi\endcsname{\def\PY@tc##1{\textcolor[rgb]{0.10,0.09,0.49}{##1}}}
\expandafter\def\csname PY@tok@vm\endcsname{\def\PY@tc##1{\textcolor[rgb]{0.10,0.09,0.49}{##1}}}
\expandafter\def\csname PY@tok@sa\endcsname{\def\PY@tc##1{\textcolor[rgb]{0.73,0.13,0.13}{##1}}}
\expandafter\def\csname PY@tok@sb\endcsname{\def\PY@tc##1{\textcolor[rgb]{0.73,0.13,0.13}{##1}}}
\expandafter\def\csname PY@tok@sc\endcsname{\def\PY@tc##1{\textcolor[rgb]{0.73,0.13,0.13}{##1}}}
\expandafter\def\csname PY@tok@dl\endcsname{\def\PY@tc##1{\textcolor[rgb]{0.73,0.13,0.13}{##1}}}
\expandafter\def\csname PY@tok@s2\endcsname{\def\PY@tc##1{\textcolor[rgb]{0.73,0.13,0.13}{##1}}}
\expandafter\def\csname PY@tok@sh\endcsname{\def\PY@tc##1{\textcolor[rgb]{0.73,0.13,0.13}{##1}}}
\expandafter\def\csname PY@tok@s1\endcsname{\def\PY@tc##1{\textcolor[rgb]{0.73,0.13,0.13}{##1}}}
\expandafter\def\csname PY@tok@mb\endcsname{\def\PY@tc##1{\textcolor[rgb]{0.40,0.40,0.40}{##1}}}
\expandafter\def\csname PY@tok@mf\endcsname{\def\PY@tc##1{\textcolor[rgb]{0.40,0.40,0.40}{##1}}}
\expandafter\def\csname PY@tok@mh\endcsname{\def\PY@tc##1{\textcolor[rgb]{0.40,0.40,0.40}{##1}}}
\expandafter\def\csname PY@tok@mi\endcsname{\def\PY@tc##1{\textcolor[rgb]{0.40,0.40,0.40}{##1}}}
\expandafter\def\csname PY@tok@il\endcsname{\def\PY@tc##1{\textcolor[rgb]{0.40,0.40,0.40}{##1}}}
\expandafter\def\csname PY@tok@mo\endcsname{\def\PY@tc##1{\textcolor[rgb]{0.40,0.40,0.40}{##1}}}
\expandafter\def\csname PY@tok@ch\endcsname{\let\PY@it=\textit\def\PY@tc##1{\textcolor[rgb]{0.25,0.50,0.50}{##1}}}
\expandafter\def\csname PY@tok@cm\endcsname{\let\PY@it=\textit\def\PY@tc##1{\textcolor[rgb]{0.25,0.50,0.50}{##1}}}
\expandafter\def\csname PY@tok@cpf\endcsname{\let\PY@it=\textit\def\PY@tc##1{\textcolor[rgb]{0.25,0.50,0.50}{##1}}}
\expandafter\def\csname PY@tok@c1\endcsname{\let\PY@it=\textit\def\PY@tc##1{\textcolor[rgb]{0.25,0.50,0.50}{##1}}}
\expandafter\def\csname PY@tok@cs\endcsname{\let\PY@it=\textit\def\PY@tc##1{\textcolor[rgb]{0.25,0.50,0.50}{##1}}}

\def\PYZbs{\char`\\}
\def\PYZus{\char`\_}
\def\PYZob{\char`\{}
\def\PYZcb{\char`\}}
\def\PYZca{\char`\^}
\def\PYZam{\char`\&}
\def\PYZlt{\char`\<}
\def\PYZgt{\char`\>}
\def\PYZsh{\char`\#}
\def\PYZpc{\char`\%}
\def\PYZdl{\char`\$}
\def\PYZhy{\char`\-}
\def\PYZsq{\char`\'}
\def\PYZdq{\char`\"}
\def\PYZti{\char`\~}
% for compatibility with earlier versions
\def\PYZat{@}
\def\PYZlb{[}
\def\PYZrb{]}
\makeatother


    % Exact colors from NB
    \definecolor{incolor}{rgb}{0.0, 0.0, 0.5}
    \definecolor{outcolor}{rgb}{0.545, 0.0, 0.0}



    
    % Prevent overflowing lines due to hard-to-break entities
    \sloppy 
    % Setup hyperref package
    \hypersetup{
      breaklinks=true,  % so long urls are correctly broken across lines
      colorlinks=true,
      urlcolor=urlcolor,
      linkcolor=linkcolor,
      citecolor=citecolor,
      }
    % Slightly bigger margins than the latex defaults
    
    \geometry{verbose,tmargin=1in,bmargin=1in,lmargin=1in,rmargin=1in}
    
    

    \begin{document}
    
    
    \maketitle
    
    

    
    \section{Felderítő adatelemzés}\label{felderuxedtux151-adatelemzuxe9s}

    \begin{Verbatim}[commandchars=\\\{\}]
{\color{incolor}In [{\color{incolor}1}]:} \PY{k+kn}{from} \PY{n+nn}{src}\PY{n+nn}{.}\PY{n+nn}{data} \PY{k}{import} \PY{n}{read\PYZus{}data}\PY{p}{,}\PY{n}{prepare\PYZus{}data}
        \PY{k+kn}{from} \PY{n+nn}{src}\PY{n+nn}{.}\PY{n+nn}{models} \PY{k}{import} \PY{n}{train\PYZus{}model}\PY{p}{,}\PY{n}{predict\PYZus{}model}
        \PY{k+kn}{import} \PY{n+nn}{pandas} \PY{k}{as} \PY{n+nn}{pd}
        \PY{k+kn}{from} \PY{n+nn}{pandas}\PY{n+nn}{.}\PY{n+nn}{plotting} \PY{k}{import} \PY{n}{lag\PYZus{}plot}
        \PY{k+kn}{from} \PY{n+nn}{pandas}\PY{n+nn}{.}\PY{n+nn}{plotting} \PY{k}{import} \PY{n}{autocorrelation\PYZus{}plot}
        \PY{k+kn}{import} \PY{n+nn}{matplotlib}\PY{n+nn}{.}\PY{n+nn}{pyplot} \PY{k}{as} \PY{n+nn}{plt}
        \PY{k+kn}{import} \PY{n+nn}{seaborn} \PY{k}{as} \PY{n+nn}{sns}
        \PY{n}{pd}\PY{o}{.}\PY{n}{options}\PY{o}{.}\PY{n}{mode}\PY{o}{.}\PY{n}{chained\PYZus{}assignment} \PY{o}{=} \PY{k+kc}{None}
        \PY{o}{\PYZpc{}}\PY{k}{matplotlib} inline
\end{Verbatim}


    Beolvastam a teljes adathalmazt és összekapcsoltam a magyarázó
változókkal

    \begin{Verbatim}[commandchars=\\\{\}]
{\color{incolor}In [{\color{incolor}2}]:} \PY{c+c1}{\PYZsh{} Reading the data}
        \PY{n}{PATH\PYZus{}TO\PYZus{}TRAIN} \PY{o}{=} \PY{l+s+s1}{\PYZsq{}}\PY{l+s+s1}{../data/raw/train15.csv}\PY{l+s+s1}{\PYZsq{}}
        \PY{n}{PATH\PYZus{}TO\PYZus{}PREDICTORS} \PY{o}{=} \PY{l+s+s1}{\PYZsq{}}\PY{l+s+s1}{../data/raw/predictors15.csv}\PY{l+s+s1}{\PYZsq{}}
        \PY{n}{INDEX\PYZus{}COLUMN} \PY{o}{=} \PY{l+s+s1}{\PYZsq{}}\PY{l+s+s1}{TIMESTAMP}\PY{l+s+s1}{\PYZsq{}}
        \PY{n}{TARGET\PYZus{}COLUMN} \PY{o}{=} \PY{l+s+s1}{\PYZsq{}}\PY{l+s+s1}{POWER}\PY{l+s+s1}{\PYZsq{}}
        \PY{n}{df\PYZus{}power} \PY{o}{=} \PY{n}{pd}\PY{o}{.}\PY{n}{read\PYZus{}csv}\PY{p}{(}\PY{n}{PATH\PYZus{}TO\PYZus{}TRAIN}\PY{p}{)}
        \PY{n}{df\PYZus{}features} \PY{o}{=} \PY{n}{pd}\PY{o}{.}\PY{n}{read\PYZus{}csv}\PY{p}{(}\PY{n}{PATH\PYZus{}TO\PYZus{}PREDICTORS}\PY{p}{)}
        \PY{n}{df\PYZus{}original} \PY{o}{=} \PY{n}{df\PYZus{}power}\PY{o}{.}\PY{n}{merge}\PY{p}{(}\PY{n}{df\PYZus{}features}\PY{p}{,} \PY{n}{how}\PY{o}{=}\PY{l+s+s1}{\PYZsq{}}\PY{l+s+s1}{left}\PY{l+s+s1}{\PYZsq{}}\PY{p}{,} \PY{n}{on}\PY{o}{=}\PY{p}{[}\PY{n}{INDEX\PYZus{}COLUMN}\PY{p}{,}\PY{l+s+s1}{\PYZsq{}}\PY{l+s+s1}{ZONEID}\PY{l+s+s1}{\PYZsq{}}\PY{p}{]}\PY{p}{)}
\end{Verbatim}


    Megnéztem, hogy néz ki az adat.

    \begin{Verbatim}[commandchars=\\\{\}]
{\color{incolor}In [{\color{incolor}3}]:} \PY{n}{df\PYZus{}original}\PY{o}{.}\PY{n}{describe}\PY{p}{(}\PY{p}{)}
\end{Verbatim}


\begin{Verbatim}[commandchars=\\\{\}]
{\color{outcolor}Out[{\color{outcolor}3}]:}              ZONEID         POWER         VAR78         VAR79        VAR134  \textbackslash{}
        count  56952.000000  56952.000000  56952.000000  56952.000000  56952.000000   
        mean       2.000000      0.186872      0.041470      0.016408  93807.236996   
        std        0.816504      0.274926      0.122040      0.050371   1106.902058   
        min        1.000000      0.000000      0.000000      0.000000  90345.875000   
        25\%        1.000000      0.000000      0.000000      0.000000  92903.000000   
        50\%        2.000000      0.003117      0.002919      0.000117  94008.093750   
        75\%        3.000000      0.333608      0.033479      0.007191  94668.203125   
        max        3.000000      1.003550      4.103422      0.820470  96317.125000   
        
                     VAR157        VAR164        VAR165        VAR166        VAR167  \textbackslash{}
        count  56952.000000  56952.000000  56952.000000  56952.000000  56952.000000   
        mean      67.610793      0.436522      0.738276     -0.257701    285.435328   
        std       20.336232      0.400015      2.452712      1.983266      7.051843   
        min        6.348233      0.000000     -8.908603     -8.404230    269.436279   
        25\%       52.359467      0.012650     -0.775085     -1.594760    280.142761   
        50\%       70.545609      0.353457      0.368076     -0.411445    284.784790   
        75\%       84.952175      0.884594      2.143609      0.944802    290.000977   
        max      102.838364      1.000008     13.056718     10.710998    310.458496   
        
                     VAR169        VAR175        VAR178        VAR228  
        count  5.695200e+04  5.695200e+04  5.695200e+04  56952.000000  
        mean   1.250545e+07  1.382187e+07  1.462089e+07      0.000997  
        std    6.593117e+06  7.755640e+06  7.049271e+06      0.003445  
        min    1.147127e+05  8.216176e+05  6.465230e+05      0.000000  
        25\%    7.616334e+06  7.271122e+06  9.568199e+06      0.000000  
        50\%    1.141026e+07  1.368581e+07  1.342308e+07      0.000000  
        75\%    1.742018e+07  2.002196e+07  2.002346e+07      0.000257  
        max    3.462558e+07  3.467366e+07  3.772200e+07      0.055961  
\end{Verbatim}
            
    \begin{Verbatim}[commandchars=\\\{\}]
{\color{incolor}In [{\color{incolor}4}]:} \PY{n}{df\PYZus{}original}\PY{o}{.}\PY{n}{head}\PY{p}{(}\PY{p}{)}
\end{Verbatim}


\begin{Verbatim}[commandchars=\\\{\}]
{\color{outcolor}Out[{\color{outcolor}4}]:}    ZONEID       TIMESTAMP     POWER     VAR78     VAR79      VAR134  \textbackslash{}
        0       1  20120401 01:00  0.754103  0.001967  0.003609  94843.6250   
        1       1  20120401 02:00  0.555000  0.005524  0.033575  94757.9375   
        2       1  20120401 03:00  0.438397  0.030113  0.132009  94732.8125   
        3       1  20120401 04:00  0.145449  0.057167  0.110645  94704.0625   
        4       1  20120401 05:00  0.111987  0.051027  0.189560  94675.0000   
        
              VAR157    VAR164    VAR165    VAR166      VAR167      VAR169     VAR175  \textbackslash{}
        0  60.221909  0.244601  1.039334 -2.503039  294.448486   2577830.0  1202532.0   
        1  54.678604  0.457138  2.482865 -2.993330  295.651367   5356093.0  2446757.0   
        2  61.294891  0.771429  3.339867 -1.982535  294.454590   7921788.0  3681336.0   
        3  67.775284  0.965866  3.106102 -1.446051  293.261475   9860520.0  4921504.0   
        4  70.172989  0.944669  2.601146 -1.904493  292.732910  11143097.0  6254380.0   
        
               VAR178    VAR228  
        0   2861797.0  0.000000  
        1   5949378.0  0.000000  
        2   8939176.0  0.001341  
        3  11331679.0  0.002501  
        4  13105558.0  0.003331  
\end{Verbatim}
            
    \begin{Verbatim}[commandchars=\\\{\}]
{\color{incolor}In [{\color{incolor}5}]:}  \PY{n}{df\PYZus{}original}\PY{o}{.}\PY{n}{info}\PY{p}{(}\PY{p}{)}
\end{Verbatim}


    \begin{Verbatim}[commandchars=\\\{\}]
<class 'pandas.core.frame.DataFrame'>
Int64Index: 56952 entries, 0 to 56951
Data columns (total 15 columns):
ZONEID       56952 non-null int64
TIMESTAMP    56952 non-null object
POWER        56952 non-null float64
VAR78        56952 non-null float64
VAR79        56952 non-null float64
VAR134       56952 non-null float64
VAR157       56952 non-null float64
VAR164       56952 non-null float64
VAR165       56952 non-null float64
VAR166       56952 non-null float64
VAR167       56952 non-null float64
VAR169       56952 non-null float64
VAR175       56952 non-null float64
VAR178       56952 non-null float64
VAR228       56952 non-null float64
dtypes: float64(13), int64(1), object(1)
memory usage: 7.0+ MB

    \end{Verbatim}

    A fentiekből kiolvasva látsztik, hogy a timestamp nem dateTime típusú,
illetve az is hogy milyen formátumban van. Ezek alapján át lehet
konvertálni. Ami még látszik belőle, hogy az egyes mérési eredmények
óránként vannak megadva.

    \begin{Verbatim}[commandchars=\\\{\}]
{\color{incolor}In [{\color{incolor}6}]:} \PY{n}{df} \PY{o}{=} \PY{n}{df\PYZus{}original}\PY{o}{.}\PY{n}{copy}\PY{p}{(}\PY{p}{)}
        \PY{n}{DATE\PYZus{}FORMAT} \PY{o}{=} \PY{l+s+s1}{\PYZsq{}}\PY{l+s+s1}{\PYZpc{}}\PY{l+s+s1}{Y}\PY{l+s+s1}{\PYZpc{}}\PY{l+s+s1}{m}\PY{l+s+si}{\PYZpc{}d}\PY{l+s+s1}{ }\PY{l+s+s1}{\PYZpc{}}\PY{l+s+s1}{H:}\PY{l+s+s1}{\PYZpc{}}\PY{l+s+s1}{M}\PY{l+s+s1}{\PYZsq{}}
        \PY{n}{dateparse} \PY{o}{=} \PY{k}{lambda} \PY{n}{x}\PY{p}{:} \PY{n}{pd}\PY{o}{.}\PY{n}{datetime}\PY{o}{.}\PY{n}{strptime}\PY{p}{(}\PY{n}{x}\PY{p}{,} \PY{n}{DATE\PYZus{}FORMAT}\PY{p}{)}
        \PY{n}{df}\PY{p}{[}\PY{n}{INDEX\PYZus{}COLUMN}\PY{p}{]} \PY{o}{=} \PY{n}{df}\PY{p}{[}\PY{n}{INDEX\PYZus{}COLUMN}\PY{p}{]}\PY{o}{.}\PY{n}{apply}\PY{p}{(}\PY{n}{dateparse}\PY{p}{)}
\end{Verbatim}


    \begin{Verbatim}[commandchars=\\\{\}]
{\color{incolor}In [{\color{incolor}7}]:} \PY{n}{df}\PY{p}{[}\PY{n}{INDEX\PYZus{}COLUMN}\PY{p}{]}\PY{o}{.}\PY{n}{head}\PY{p}{(}\PY{l+m+mi}{1}\PY{p}{)}
\end{Verbatim}


\begin{Verbatim}[commandchars=\\\{\}]
{\color{outcolor}Out[{\color{outcolor}7}]:} 0   2012-04-01 01:00:00
        Name: TIMESTAMP, dtype: datetime64[ns]
\end{Verbatim}
            
    Amit még érdemes átalakítani , az a feature változók elnevezése. VARXY
elég nehezen értelmezhető, ezért átneveztem őket a feladaleírásban
szereplő nevekkel.

    \begin{Verbatim}[commandchars=\\\{\}]
{\color{incolor}In [{\color{incolor}8}]:} \PY{n}{column\PYZus{}mapping} \PY{o}{=} \PY{p}{\PYZob{}}\PY{l+s+s2}{\PYZdq{}}\PY{l+s+s2}{VAR78}\PY{l+s+s2}{\PYZdq{}}\PY{p}{:}\PY{l+s+s2}{\PYZdq{}}\PY{l+s+s2}{LIQUID\PYZus{}WATER}\PY{l+s+s2}{\PYZdq{}}\PY{p}{,} \PY{l+s+s2}{\PYZdq{}}\PY{l+s+s2}{VAR79}\PY{l+s+s2}{\PYZdq{}}\PY{p}{:} \PY{l+s+s2}{\PYZdq{}}\PY{l+s+s2}{ICE\PYZus{}WATER}\PY{l+s+s2}{\PYZdq{}}\PY{p}{,}  \PY{l+s+s2}{\PYZdq{}}\PY{l+s+s2}{VAR134}\PY{l+s+s2}{\PYZdq{}}\PY{p}{:}\PY{l+s+s2}{\PYZdq{}}\PY{l+s+s2}{SURFACE\PYZus{}PRESSURE}\PY{l+s+s2}{\PYZdq{}}\PY{p}{,}\PY{l+s+s2}{\PYZdq{}}\PY{l+s+s2}{VAR157}\PY{l+s+s2}{\PYZdq{}}\PY{p}{:}\PY{l+s+s2}{\PYZdq{}}\PY{l+s+s2}{RELATIVE\PYZus{}HUMIDITY}\PY{l+s+s2}{\PYZdq{}}\PY{p}{,}\PY{l+s+s2}{\PYZdq{}}\PY{l+s+s2}{VAR164}\PY{l+s+s2}{\PYZdq{}}\PY{p}{:}\PY{l+s+s2}{\PYZdq{}}\PY{l+s+s2}{TOTAL\PYZus{}CLOUD\PYZus{}COVER}\PY{l+s+s2}{\PYZdq{}}\PY{p}{,}\PY{l+s+s2}{\PYZdq{}}\PY{l+s+s2}{VAR165}\PY{l+s+s2}{\PYZdq{}}\PY{p}{:}\PY{l+s+s2}{\PYZdq{}}\PY{l+s+s2}{WIND\PYZus{}U}\PY{l+s+s2}{\PYZdq{}}\PY{p}{,}\PY{l+s+s2}{\PYZdq{}}\PY{l+s+s2}{VAR166}\PY{l+s+s2}{\PYZdq{}}\PY{p}{:}\PY{l+s+s2}{\PYZdq{}}\PY{l+s+s2}{WIND\PYZus{}V}\PY{l+s+s2}{\PYZdq{}}\PY{p}{,}\PY{l+s+s2}{\PYZdq{}}\PY{l+s+s2}{VAR167}\PY{l+s+s2}{\PYZdq{}}\PY{p}{:}\PY{l+s+s2}{\PYZdq{}}\PY{l+s+s2}{TEMPERATURE}\PY{l+s+s2}{\PYZdq{}}\PY{p}{,}\PY{l+s+s2}{\PYZdq{}}\PY{l+s+s2}{VAR169}\PY{l+s+s2}{\PYZdq{}}\PY{p}{:}\PY{l+s+s2}{\PYZdq{}}\PY{l+s+s2}{SOLAR\PYZus{}RAD}\PY{l+s+s2}{\PYZdq{}}\PY{p}{,}\PY{l+s+s2}{\PYZdq{}}\PY{l+s+s2}{VAR175}\PY{l+s+s2}{\PYZdq{}}\PY{p}{:}\PY{l+s+s2}{\PYZdq{}}\PY{l+s+s2}{TERMAL\PYZus{}RAD}\PY{l+s+s2}{\PYZdq{}}\PY{p}{,}\PY{l+s+s2}{\PYZdq{}}\PY{l+s+s2}{VAR178}\PY{l+s+s2}{\PYZdq{}}\PY{p}{:}\PY{l+s+s2}{\PYZdq{}}\PY{l+s+s2}{TOP\PYZus{}NET\PYZus{}SOLAR\PYZus{}RAD}\PY{l+s+s2}{\PYZdq{}}\PY{p}{,}\PY{l+s+s2}{\PYZdq{}}\PY{l+s+s2}{VAR228}\PY{l+s+s2}{\PYZdq{}}\PY{p}{:}\PY{l+s+s2}{\PYZdq{}}\PY{l+s+s2}{TOTAL\PYZus{}PRECIPATION}\PY{l+s+s2}{\PYZdq{}}\PY{p}{\PYZcb{}}
        \PY{n}{df} \PY{o}{=} \PY{n}{df}\PY{o}{.}\PY{n}{rename}\PY{p}{(}\PY{n}{index}\PY{o}{=}\PY{n+nb}{str}\PY{p}{,} \PY{n}{columns}\PY{o}{=}\PY{n}{column\PYZus{}mapping}\PY{p}{)}
        \PY{n+nb}{print}\PY{p}{(}\PY{n}{df}\PY{o}{.}\PY{n}{columns}\PY{p}{)}
\end{Verbatim}


    \begin{Verbatim}[commandchars=\\\{\}]
Index(['ZONEID', 'TIMESTAMP', 'POWER', 'LIQUID\_WATER', 'ICE\_WATER',
       'SURFACE\_PRESSURE', 'RELATIVE\_HUMIDITY', 'TOTAL\_CLOUD\_COVER', 'WIND\_U',
       'WIND\_V', 'TEMPERATURE', 'SOLAR\_RAD', 'TERMAL\_RAD', 'TOP\_NET\_SOLAR\_RAD',
       'TOTAL\_PRECIPATION'],
      dtype='object')

    \end{Verbatim}

    Megnéztem zónánként milyen az eloszlása a célváltozónak

    \begin{Verbatim}[commandchars=\\\{\}]
{\color{incolor}In [{\color{incolor}9}]:} \PY{n}{g} \PY{o}{=} \PY{n}{sns}\PY{o}{.}\PY{n}{FacetGrid}\PY{p}{(}\PY{n}{df}\PY{p}{,} \PY{n}{col}\PY{o}{=}\PY{l+s+s2}{\PYZdq{}}\PY{l+s+s2}{ZONEID}\PY{l+s+s2}{\PYZdq{}}\PY{p}{)}
        \PY{n}{g}\PY{o}{.}\PY{n}{map}\PY{p}{(}\PY{n}{sns}\PY{o}{.}\PY{n}{kdeplot}\PY{p}{,} \PY{n}{TARGET\PYZus{}COLUMN}\PY{p}{)}
\end{Verbatim}


\begin{Verbatim}[commandchars=\\\{\}]
{\color{outcolor}Out[{\color{outcolor}9}]:} <seaborn.axisgrid.FacetGrid at 0xc4a48c0b38>
\end{Verbatim}
            
    \begin{center}
    \adjustimage{max size={0.9\linewidth}{0.9\paperheight}}{output_14_1.png}
    \end{center}
    { \hspace*{\fill} \\}
    
    Mivel nagyjából egyformának tűnik, azt leszámítva, hogy az első zónában
kicsit több érték van a 0 közelében, így úgy döntöttem az elemzési
részben elég csak egyet vizsgálnom közülük első körben. Modell építésnél
is szerintem külön modellt fogok rájuk építeni.

    \begin{Verbatim}[commandchars=\\\{\}]
{\color{incolor}In [{\color{incolor}10}]:} \PY{n}{df} \PY{o}{=} \PY{n}{df}\PY{p}{[}\PY{n}{df}\PY{o}{.}\PY{n}{ZONEID} \PY{o}{==} \PY{l+m+mi}{1}\PY{p}{]}\PY{o}{.}\PY{n}{drop}\PY{p}{(}\PY{l+s+s1}{\PYZsq{}}\PY{l+s+s1}{ZONEID}\PY{l+s+s1}{\PYZsq{}}\PY{p}{,}\PY{n}{axis}\PY{o}{=}\PY{l+m+mi}{1}\PY{p}{)}
         \PY{c+c1}{\PYZsh{}df2 = df[df.ZONEID == 2].drop(\PYZsq{}ZONEID\PYZsq{},axis=1)}
         \PY{c+c1}{\PYZsh{}df3 = df[df.ZONEID == 3].drop(\PYZsq{}ZONEID\PYZsq{},axis=1)}
\end{Verbatim}


    A továbbiakban felveszem a timestampet indexnek hogy a megjelítésnél
jobban látszon a célváltozó időbeli változása.

    \begin{Verbatim}[commandchars=\\\{\}]
{\color{incolor}In [{\color{incolor}11}]:} \PY{n}{df} \PY{o}{=} \PY{n}{df}\PY{o}{.}\PY{n}{set\PYZus{}index}\PY{p}{(}\PY{n}{INDEX\PYZus{}COLUMN}\PY{p}{)}
         \PY{c+c1}{\PYZsh{} df[\PYZdq{}MONTH\PYZdq{}] = df.index.to\PYZus{}series().apply(lambda x: x.month)}
         \PY{c+c1}{\PYZsh{} df[\PYZdq{}HOUR\PYZdq{}] = df.index.to\PYZus{}series().apply(lambda x: x.hour)}
\end{Verbatim}


    Mivel óránkénti adatunk van, ezért az alábbi módon tudjuk definiálni, a
hónapot, hetet, évet.

    \begin{Verbatim}[commandchars=\\\{\}]
{\color{incolor}In [{\color{incolor}12}]:} \PY{n}{ONE\PYZus{}DAY} \PY{o}{=} \PY{l+m+mi}{24}
         \PY{n}{ONE\PYZus{}WEEK} \PY{o}{=} \PY{l+m+mi}{7} \PY{o}{*} \PY{n}{ONE\PYZus{}DAY}
         \PY{n}{ONE\PYZus{}MONTH} \PY{o}{=} \PY{l+m+mi}{30} \PY{o}{*} \PY{n}{ONE\PYZus{}DAY}
         \PY{n}{ONE\PYZus{}YEAR} \PY{o}{=} \PY{l+m+mi}{365} \PY{o}{*} \PY{n}{ONE\PYZus{}DAY}
\end{Verbatim}


    Megnéztem napi, illetve heti szinten hogyan változik a célváltozó, ebből
az látszik, hogy van egy napi szintű szezonalitása a célváltozónak

    \begin{Verbatim}[commandchars=\\\{\}]
{\color{incolor}In [{\color{incolor}13}]:} \PY{n}{fig}\PY{p}{,} \PY{n}{ax} \PY{o}{=} \PY{n}{plt}\PY{o}{.}\PY{n}{subplots}\PY{p}{(}\PY{l+m+mi}{1}\PY{p}{,}\PY{l+m+mi}{2}\PY{p}{,}\PY{n}{figsize}\PY{o}{=}\PY{p}{(}\PY{l+m+mi}{20}\PY{p}{,}\PY{l+m+mi}{10}\PY{p}{)}\PY{p}{)}
         \PY{n}{df}\PY{p}{[}\PY{p}{:}\PY{l+m+mi}{2}\PY{o}{*}\PY{n}{ONE\PYZus{}DAY}\PY{p}{]}\PY{o}{.}\PY{n}{plot}\PY{p}{(}\PY{n}{y}\PY{o}{=}\PY{n}{TARGET\PYZus{}COLUMN}\PY{p}{,}\PY{n}{ax}\PY{o}{=}\PY{n}{ax}\PY{p}{[}\PY{l+m+mi}{0}\PY{p}{]}\PY{p}{)}
         \PY{n}{df}\PY{p}{[}\PY{p}{:}\PY{n}{ONE\PYZus{}WEEK}\PY{p}{]}\PY{o}{.}\PY{n}{plot}\PY{p}{(}\PY{n}{y}\PY{o}{=}\PY{n}{TARGET\PYZus{}COLUMN}\PY{p}{,}\PY{n}{figsize}\PY{o}{=}\PY{p}{(}\PY{l+m+mi}{20}\PY{p}{,}\PY{l+m+mi}{5}\PY{p}{)}\PY{p}{,}\PY{n}{ax}\PY{o}{=}\PY{n}{ax}\PY{p}{[}\PY{l+m+mi}{1}\PY{p}{]}\PY{p}{)}
\end{Verbatim}


\begin{Verbatim}[commandchars=\\\{\}]
{\color{outcolor}Out[{\color{outcolor}13}]:} <matplotlib.axes.\_subplots.AxesSubplot at 0xc4a3d98c50>
\end{Verbatim}
            
    \begin{center}
    \adjustimage{max size={0.9\linewidth}{0.9\paperheight}}{output_22_1.png}
    \end{center}
    { \hspace*{\fill} \\}
    
    Ez után megnéztem, ha más felbontásban nézem az adatot (napi/havi/fél
éves), akkor milyen mintát lehet benne felfedezni.

    \begin{Verbatim}[commandchars=\\\{\}]
{\color{incolor}In [{\color{incolor}14}]:} \PY{n}{power\PYZus{}by\PYZus{}day} \PY{o}{=} \PY{n}{df}\PY{o}{.}\PY{n}{resample}\PY{p}{(}\PY{l+s+s1}{\PYZsq{}}\PY{l+s+s1}{D}\PY{l+s+s1}{\PYZsq{}}\PY{p}{)}
         \PY{n}{power\PYZus{}by\PYZus{}month} \PY{o}{=} \PY{n}{df}\PY{o}{.}\PY{n}{resample}\PY{p}{(}\PY{l+s+s1}{\PYZsq{}}\PY{l+s+s1}{M}\PY{l+s+s1}{\PYZsq{}}\PY{p}{)}
         \PY{n}{power\PYZus{}by\PYZus{}semester} \PY{o}{=} \PY{n}{df}\PY{o}{.}\PY{n}{resample}\PY{p}{(}\PY{l+s+s1}{\PYZsq{}}\PY{l+s+s1}{6M}\PY{l+s+s1}{\PYZsq{}}\PY{p}{)}
         
         \PY{n}{fig}\PY{p}{,} \PY{n}{ax} \PY{o}{=} \PY{n}{plt}\PY{o}{.}\PY{n}{subplots}\PY{p}{(}\PY{l+m+mi}{3}\PY{p}{,}\PY{l+m+mi}{2}\PY{p}{,}\PY{n}{figsize}\PY{o}{=}\PY{p}{(}\PY{l+m+mi}{20}\PY{p}{,}\PY{l+m+mi}{10}\PY{p}{)}\PY{p}{)}
         
         \PY{n}{power\PYZus{}by\PYZus{}day}\PY{o}{.}\PY{n}{mean}\PY{p}{(}\PY{p}{)}\PY{o}{.}\PY{n}{plot}\PY{p}{(}\PY{n}{y}\PY{o}{=}\PY{l+s+s1}{\PYZsq{}}\PY{l+s+s1}{POWER}\PY{l+s+s1}{\PYZsq{}}\PY{p}{,}\PY{n}{ax} \PY{o}{=} \PY{n}{ax}\PY{p}{[}\PY{l+m+mi}{0}\PY{p}{]}\PY{p}{[}\PY{l+m+mi}{0}\PY{p}{]}\PY{p}{)}
         \PY{n}{power\PYZus{}by\PYZus{}day}\PY{o}{.}\PY{n}{median}\PY{p}{(}\PY{p}{)}\PY{o}{.}\PY{n}{plot}\PY{p}{(}\PY{n}{y}\PY{o}{=}\PY{l+s+s1}{\PYZsq{}}\PY{l+s+s1}{POWER}\PY{l+s+s1}{\PYZsq{}}\PY{p}{,} \PY{n}{ax} \PY{o}{=}\PY{n}{ax}\PY{p}{[}\PY{l+m+mi}{0}\PY{p}{]}\PY{p}{[}\PY{l+m+mi}{1}\PY{p}{]}\PY{p}{)}
         \PY{n}{power\PYZus{}by\PYZus{}month}\PY{o}{.}\PY{n}{mean}\PY{p}{(}\PY{p}{)}\PY{o}{.}\PY{n}{plot}\PY{p}{(}\PY{n}{y}\PY{o}{=}\PY{l+s+s1}{\PYZsq{}}\PY{l+s+s1}{POWER}\PY{l+s+s1}{\PYZsq{}}\PY{p}{,} \PY{n}{ax} \PY{o}{=}\PY{n}{ax}\PY{p}{[}\PY{l+m+mi}{1}\PY{p}{]}\PY{p}{[}\PY{l+m+mi}{0}\PY{p}{]}\PY{p}{)}
         \PY{n}{power\PYZus{}by\PYZus{}month}\PY{o}{.}\PY{n}{median}\PY{p}{(}\PY{p}{)}\PY{o}{.}\PY{n}{plot}\PY{p}{(}\PY{n}{y}\PY{o}{=}\PY{l+s+s1}{\PYZsq{}}\PY{l+s+s1}{POWER}\PY{l+s+s1}{\PYZsq{}}\PY{p}{,} \PY{n}{ax} \PY{o}{=}\PY{n}{ax}\PY{p}{[}\PY{l+m+mi}{1}\PY{p}{]}\PY{p}{[}\PY{l+m+mi}{1}\PY{p}{]}\PY{p}{)}
         \PY{n}{power\PYZus{}by\PYZus{}semester}\PY{o}{.}\PY{n}{mean}\PY{p}{(}\PY{p}{)}\PY{o}{.}\PY{n}{plot}\PY{p}{(}\PY{n}{y}\PY{o}{=}\PY{l+s+s1}{\PYZsq{}}\PY{l+s+s1}{POWER}\PY{l+s+s1}{\PYZsq{}}\PY{p}{,} \PY{n}{ax} \PY{o}{=}\PY{n}{ax}\PY{p}{[}\PY{l+m+mi}{2}\PY{p}{]}\PY{p}{[}\PY{l+m+mi}{0}\PY{p}{]}\PY{p}{)}
         \PY{n}{power\PYZus{}by\PYZus{}semester}\PY{o}{.}\PY{n}{median}\PY{p}{(}\PY{p}{)}\PY{o}{.}\PY{n}{plot}\PY{p}{(}\PY{n}{y}\PY{o}{=}\PY{l+s+s1}{\PYZsq{}}\PY{l+s+s1}{POWER}\PY{l+s+s1}{\PYZsq{}}\PY{p}{,} \PY{n}{ax} \PY{o}{=}\PY{n}{ax}\PY{p}{[}\PY{l+m+mi}{2}\PY{p}{]}\PY{p}{[}\PY{l+m+mi}{1}\PY{p}{]}\PY{p}{)}
\end{Verbatim}


\begin{Verbatim}[commandchars=\\\{\}]
{\color{outcolor}Out[{\color{outcolor}14}]:} <matplotlib.axes.\_subplots.AxesSubplot at 0xc4a5b895f8>
\end{Verbatim}
            
    \begin{center}
    \adjustimage{max size={0.9\linewidth}{0.9\paperheight}}{output_24_1.png}
    \end{center}
    { \hspace*{\fill} \\}
    
    Innen az látszik, hogy a napi szezonalitáson kívül, van egy éves
szezonalitása is. Ez elég magától értetődő, ha jobban belegondolunk,
mivel napelemek termelését nézzük, ez főleg a napsugárzástól függ. Így a
napi szezonalitás magyarázható azzal, hogy van este és nappal, az éves
pedig azzal hogy télen kevesebbet és gyengébben süt a nap, mint nyáron.

    A következőekben az adat stacionárius jellegát vizsgálom, adott
felbontásokra.

    \begin{Verbatim}[commandchars=\\\{\}]
{\color{incolor}In [{\color{incolor}15}]:} \PY{n}{fig}\PY{p}{,} \PY{n}{ax} \PY{o}{=} \PY{n}{plt}\PY{o}{.}\PY{n}{subplots}\PY{p}{(}\PY{l+m+mi}{2}\PY{p}{,}\PY{l+m+mi}{2}\PY{p}{,}\PY{n}{figsize}\PY{o}{=}\PY{p}{(}\PY{l+m+mi}{10}\PY{p}{,}\PY{l+m+mi}{5}\PY{p}{)}\PY{p}{)}
         \PY{c+c1}{\PYZsh{}power\PYZus{}by\PYZus{}day.sum()[\PYZdq{}POWER\PYZdq{}].plot()}
         \PY{n}{df}\PY{o}{.}\PY{n}{POWER}\PY{p}{[}\PY{n}{ONE\PYZus{}YEAR}\PY{p}{:}\PY{p}{]}\PY{o}{.}\PY{n}{plot}\PY{p}{(}\PY{n}{ax}\PY{o}{=}\PY{n}{ax}\PY{p}{[}\PY{l+m+mi}{0}\PY{p}{]}\PY{p}{[}\PY{l+m+mi}{0}\PY{p}{]}\PY{p}{)}
         \PY{n}{df}\PY{o}{.}\PY{n}{POWER}\PY{o}{.}\PY{n}{rolling}\PY{p}{(}\PY{n}{window}\PY{o}{=}\PY{n}{ONE\PYZus{}YEAR}\PY{p}{)}\PY{o}{.}\PY{n}{mean}\PY{p}{(}\PY{p}{)}\PY{p}{[}\PY{n}{ONE\PYZus{}YEAR}\PY{p}{:}\PY{p}{]}\PY{o}{.}\PY{n}{plot}\PY{p}{(}\PY{n}{color}\PY{o}{=}\PY{l+s+s2}{\PYZdq{}}\PY{l+s+s2}{orange}\PY{l+s+s2}{\PYZdq{}}\PY{p}{,}\PY{n}{ax}\PY{o}{=}\PY{n}{ax}\PY{p}{[}\PY{l+m+mi}{0}\PY{p}{]}\PY{p}{[}\PY{l+m+mi}{0}\PY{p}{]}\PY{p}{)}
         \PY{n}{df}\PY{o}{.}\PY{n}{POWER}\PY{o}{.}\PY{n}{rolling}\PY{p}{(}\PY{n}{window}\PY{o}{=}\PY{n}{ONE\PYZus{}YEAR}\PY{p}{)}\PY{o}{.}\PY{n}{std}\PY{p}{(}\PY{p}{)}\PY{p}{[}\PY{n}{ONE\PYZus{}YEAR}\PY{p}{:}\PY{p}{]}\PY{o}{.}\PY{n}{plot}\PY{p}{(}\PY{n}{color}\PY{o}{=}\PY{l+s+s2}{\PYZdq{}}\PY{l+s+s2}{black}\PY{l+s+s2}{\PYZdq{}}\PY{p}{,}\PY{n}{ax}\PY{o}{=}\PY{n}{ax}\PY{p}{[}\PY{l+m+mi}{0}\PY{p}{]}\PY{p}{[}\PY{l+m+mi}{0}\PY{p}{]}\PY{p}{)}
         
         \PY{n}{df}\PY{o}{.}\PY{n}{POWER}\PY{p}{[}\PY{n}{ONE\PYZus{}MONTH}\PY{p}{:}\PY{n}{ONE\PYZus{}YEAR}\PY{p}{]}\PY{o}{.}\PY{n}{plot}\PY{p}{(}\PY{n}{ax}\PY{o}{=}\PY{n}{ax}\PY{p}{[}\PY{l+m+mi}{0}\PY{p}{]}\PY{p}{[}\PY{l+m+mi}{1}\PY{p}{]}\PY{p}{)}
         \PY{n}{df}\PY{o}{.}\PY{n}{POWER}\PY{o}{.}\PY{n}{rolling}\PY{p}{(}\PY{n}{window}\PY{o}{=}\PY{n}{ONE\PYZus{}MONTH}\PY{p}{)}\PY{o}{.}\PY{n}{mean}\PY{p}{(}\PY{p}{)}\PY{p}{[}\PY{n}{ONE\PYZus{}MONTH}\PY{p}{:}\PY{n}{ONE\PYZus{}YEAR}\PY{p}{]}\PY{o}{.}\PY{n}{plot}\PY{p}{(}\PY{n}{color}\PY{o}{=}\PY{l+s+s2}{\PYZdq{}}\PY{l+s+s2}{orange}\PY{l+s+s2}{\PYZdq{}}\PY{p}{,}\PY{n}{ax}\PY{o}{=}\PY{n}{ax}\PY{p}{[}\PY{l+m+mi}{0}\PY{p}{]}\PY{p}{[}\PY{l+m+mi}{1}\PY{p}{]}\PY{p}{)}
         \PY{n}{df}\PY{o}{.}\PY{n}{POWER}\PY{o}{.}\PY{n}{rolling}\PY{p}{(}\PY{n}{window}\PY{o}{=}\PY{n}{ONE\PYZus{}MONTH}\PY{p}{)}\PY{o}{.}\PY{n}{std}\PY{p}{(}\PY{p}{)}\PY{p}{[}\PY{n}{ONE\PYZus{}MONTH}\PY{p}{:}\PY{n}{ONE\PYZus{}YEAR}\PY{p}{]}\PY{o}{.}\PY{n}{plot}\PY{p}{(}\PY{n}{color}\PY{o}{=}\PY{l+s+s2}{\PYZdq{}}\PY{l+s+s2}{black}\PY{l+s+s2}{\PYZdq{}}\PY{p}{,}\PY{n}{ax}\PY{o}{=}\PY{n}{ax}\PY{p}{[}\PY{l+m+mi}{0}\PY{p}{]}\PY{p}{[}\PY{l+m+mi}{1}\PY{p}{]}\PY{p}{)}
         
         \PY{n}{df}\PY{o}{.}\PY{n}{POWER}\PY{p}{[}\PY{n}{ONE\PYZus{}WEEK}\PY{p}{:}\PY{n}{ONE\PYZus{}MONTH}\PY{p}{]}\PY{o}{.}\PY{n}{plot}\PY{p}{(}\PY{n}{ax}\PY{o}{=}\PY{n}{ax}\PY{p}{[}\PY{l+m+mi}{1}\PY{p}{]}\PY{p}{[}\PY{l+m+mi}{0}\PY{p}{]}\PY{p}{)}
         \PY{n}{df}\PY{o}{.}\PY{n}{POWER}\PY{o}{.}\PY{n}{rolling}\PY{p}{(}\PY{n}{window}\PY{o}{=}\PY{n}{ONE\PYZus{}WEEK}\PY{p}{)}\PY{o}{.}\PY{n}{mean}\PY{p}{(}\PY{p}{)}\PY{p}{[}\PY{n}{ONE\PYZus{}WEEK}\PY{p}{:}\PY{n}{ONE\PYZus{}MONTH}\PY{p}{]}\PY{o}{.}\PY{n}{plot}\PY{p}{(}\PY{n}{color}\PY{o}{=}\PY{l+s+s2}{\PYZdq{}}\PY{l+s+s2}{orange}\PY{l+s+s2}{\PYZdq{}}\PY{p}{,}\PY{n}{ax}\PY{o}{=}\PY{n}{ax}\PY{p}{[}\PY{l+m+mi}{1}\PY{p}{]}\PY{p}{[}\PY{l+m+mi}{0}\PY{p}{]}\PY{p}{)}
         \PY{n}{df}\PY{o}{.}\PY{n}{POWER}\PY{o}{.}\PY{n}{rolling}\PY{p}{(}\PY{n}{window}\PY{o}{=}\PY{n}{ONE\PYZus{}WEEK}\PY{p}{)}\PY{o}{.}\PY{n}{std}\PY{p}{(}\PY{p}{)}\PY{p}{[}\PY{n}{ONE\PYZus{}WEEK}\PY{p}{:}\PY{n}{ONE\PYZus{}MONTH}\PY{p}{]}\PY{o}{.}\PY{n}{plot}\PY{p}{(}\PY{n}{color}\PY{o}{=}\PY{l+s+s2}{\PYZdq{}}\PY{l+s+s2}{black}\PY{l+s+s2}{\PYZdq{}}\PY{p}{,}\PY{n}{ax}\PY{o}{=}\PY{n}{ax}\PY{p}{[}\PY{l+m+mi}{1}\PY{p}{]}\PY{p}{[}\PY{l+m+mi}{0}\PY{p}{]}\PY{p}{)}
         
         \PY{n}{df}\PY{o}{.}\PY{n}{POWER}\PY{p}{[}\PY{n}{ONE\PYZus{}DAY}\PY{p}{:}\PY{n}{ONE\PYZus{}WEEK}\PY{p}{]}\PY{o}{.}\PY{n}{plot}\PY{p}{(}\PY{n}{ax}\PY{o}{=}\PY{n}{ax}\PY{p}{[}\PY{l+m+mi}{1}\PY{p}{]}\PY{p}{[}\PY{l+m+mi}{1}\PY{p}{]}\PY{p}{)}
         \PY{n}{df}\PY{o}{.}\PY{n}{POWER}\PY{o}{.}\PY{n}{rolling}\PY{p}{(}\PY{n}{window}\PY{o}{=}\PY{n}{ONE\PYZus{}DAY}\PY{p}{)}\PY{o}{.}\PY{n}{mean}\PY{p}{(}\PY{p}{)}\PY{p}{[}\PY{n}{ONE\PYZus{}DAY}\PY{p}{:}\PY{n}{ONE\PYZus{}WEEK}\PY{p}{]}\PY{o}{.}\PY{n}{plot}\PY{p}{(}\PY{n}{color}\PY{o}{=}\PY{l+s+s2}{\PYZdq{}}\PY{l+s+s2}{orange}\PY{l+s+s2}{\PYZdq{}}\PY{p}{,}\PY{n}{ax}\PY{o}{=}\PY{n}{ax}\PY{p}{[}\PY{l+m+mi}{1}\PY{p}{]}\PY{p}{[}\PY{l+m+mi}{1}\PY{p}{]}\PY{p}{)}
         \PY{n}{df}\PY{o}{.}\PY{n}{POWER}\PY{o}{.}\PY{n}{rolling}\PY{p}{(}\PY{n}{window}\PY{o}{=}\PY{n}{ONE\PYZus{}DAY}\PY{p}{)}\PY{o}{.}\PY{n}{std}\PY{p}{(}\PY{p}{)}\PY{p}{[}\PY{n}{ONE\PYZus{}DAY}\PY{p}{:}\PY{n}{ONE\PYZus{}WEEK}\PY{p}{]}\PY{o}{.}\PY{n}{plot}\PY{p}{(}\PY{n}{color}\PY{o}{=}\PY{l+s+s2}{\PYZdq{}}\PY{l+s+s2}{black}\PY{l+s+s2}{\PYZdq{}}\PY{p}{,}\PY{n}{ax}\PY{o}{=}\PY{n}{ax}\PY{p}{[}\PY{l+m+mi}{1}\PY{p}{]}\PY{p}{[}\PY{l+m+mi}{1}\PY{p}{]}\PY{p}{)}
\end{Verbatim}


\begin{Verbatim}[commandchars=\\\{\}]
{\color{outcolor}Out[{\color{outcolor}15}]:} <matplotlib.axes.\_subplots.AxesSubplot at 0xc4a699cd30>
\end{Verbatim}
            
    \begin{center}
    \adjustimage{max size={0.9\linewidth}{0.9\paperheight}}{output_27_1.png}
    \end{center}
    { \hspace*{\fill} \\}
    
    Éves szinten nézve stacionárius, más felbontásban kevésbé.

    Ezután megvizsgáltam, hogy statisztikailag milyen komponensekből áll az
adat.

    \begin{Verbatim}[commandchars=\\\{\}]
{\color{incolor}In [{\color{incolor}16}]:} \PY{k+kn}{from} \PY{n+nn}{statsmodels}\PY{n+nn}{.}\PY{n+nn}{tsa}\PY{n+nn}{.}\PY{n+nn}{seasonal} \PY{k}{import} \PY{n}{seasonal\PYZus{}decompose}
         \PY{n}{decomposition} \PY{o}{=} \PY{n}{seasonal\PYZus{}decompose}\PY{p}{(}\PY{n}{power\PYZus{}by\PYZus{}month}\PY{o}{.}\PY{n}{sum}\PY{p}{(}\PY{p}{)}\PY{o}{.}\PY{n}{POWER}\PY{p}{)}
         \PY{n}{trend} \PY{o}{=} \PY{n}{decomposition}\PY{o}{.}\PY{n}{trend}
         \PY{n}{seasonality} \PY{o}{=} \PY{n}{decomposition}\PY{o}{.}\PY{n}{seasonal}
         \PY{n}{residual} \PY{o}{=} \PY{n}{decomposition}\PY{o}{.}\PY{n}{resid}
         
         
         \PY{n}{fig}\PY{p}{,} \PY{n}{ax} \PY{o}{=} \PY{n}{plt}\PY{o}{.}\PY{n}{subplots}\PY{p}{(}\PY{n}{figsize} \PY{o}{=} \PY{p}{(}\PY{l+m+mi}{20}\PY{p}{,}\PY{l+m+mi}{2}\PY{p}{)}\PY{p}{)}
         \PY{n}{trend}\PY{p}{[}\PY{p}{:}\PY{l+m+mi}{200}\PY{p}{]}\PY{o}{.}\PY{n}{plot}\PY{p}{(}\PY{n}{ax}\PY{o}{=}\PY{n}{ax}\PY{p}{)}
         \PY{n}{fig}\PY{p}{,} \PY{n}{ax} \PY{o}{=} \PY{n}{plt}\PY{o}{.}\PY{n}{subplots}\PY{p}{(}\PY{l+m+mi}{1}\PY{p}{,}\PY{l+m+mi}{2}\PY{p}{,}\PY{n}{figsize} \PY{o}{=} \PY{p}{(}\PY{l+m+mi}{20}\PY{p}{,}\PY{l+m+mi}{2}\PY{p}{)}\PY{p}{)}
         \PY{n}{seasonality}\PY{p}{[}\PY{p}{:}\PY{l+m+mi}{200}\PY{p}{]}\PY{o}{.}\PY{n}{plot}\PY{p}{(}\PY{n}{ax}\PY{o}{=}\PY{n}{ax}\PY{p}{[}\PY{l+m+mi}{0}\PY{p}{]}\PY{p}{)}
         \PY{n}{residual}\PY{p}{[}\PY{p}{:}\PY{l+m+mi}{200}\PY{p}{]}\PY{o}{.}\PY{n}{plot}\PY{p}{(}\PY{n}{ax}\PY{o}{=}\PY{n}{ax}\PY{p}{[}\PY{l+m+mi}{1}\PY{p}{]}\PY{p}{)}
         
         \PY{n}{decomposition} \PY{o}{=} \PY{n}{seasonal\PYZus{}decompose}\PY{p}{(}\PY{n}{power\PYZus{}by\PYZus{}day}\PY{o}{.}\PY{n}{sum}\PY{p}{(}\PY{p}{)}\PY{o}{.}\PY{n}{POWER}\PY{p}{)}
         \PY{n}{trend} \PY{o}{=} \PY{n}{decomposition}\PY{o}{.}\PY{n}{trend}
         \PY{n}{seasonality} \PY{o}{=} \PY{n}{decomposition}\PY{o}{.}\PY{n}{seasonal}
         \PY{n}{residual} \PY{o}{=} \PY{n}{decomposition}\PY{o}{.}\PY{n}{resid}
         
         
         \PY{n}{fig}\PY{p}{,} \PY{n}{ax} \PY{o}{=} \PY{n}{plt}\PY{o}{.}\PY{n}{subplots}\PY{p}{(}\PY{n}{figsize} \PY{o}{=} \PY{p}{(}\PY{l+m+mi}{20}\PY{p}{,}\PY{l+m+mi}{2}\PY{p}{)}\PY{p}{)}
         \PY{n}{trend}\PY{p}{[}\PY{p}{:}\PY{l+m+mi}{200}\PY{p}{]}\PY{o}{.}\PY{n}{plot}\PY{p}{(}\PY{n}{ax}\PY{o}{=}\PY{n}{ax}\PY{p}{)}
         \PY{n}{fig}\PY{p}{,} \PY{n}{ax} \PY{o}{=} \PY{n}{plt}\PY{o}{.}\PY{n}{subplots}\PY{p}{(}\PY{l+m+mi}{1}\PY{p}{,}\PY{l+m+mi}{2}\PY{p}{,}\PY{n}{figsize} \PY{o}{=} \PY{p}{(}\PY{l+m+mi}{20}\PY{p}{,}\PY{l+m+mi}{2}\PY{p}{)}\PY{p}{)}
         \PY{n}{seasonality}\PY{p}{[}\PY{p}{:}\PY{l+m+mi}{200}\PY{p}{]}\PY{o}{.}\PY{n}{plot}\PY{p}{(}\PY{n}{ax}\PY{o}{=}\PY{n}{ax}\PY{p}{[}\PY{l+m+mi}{0}\PY{p}{]}\PY{p}{)}
         \PY{n}{residual}\PY{p}{[}\PY{p}{:}\PY{l+m+mi}{200}\PY{p}{]}\PY{o}{.}\PY{n}{plot}\PY{p}{(}\PY{n}{ax}\PY{o}{=}\PY{n}{ax}\PY{p}{[}\PY{l+m+mi}{1}\PY{p}{]}\PY{p}{)}
         
         
         \PY{n}{decomposition} \PY{o}{=} \PY{n}{seasonal\PYZus{}decompose}\PY{p}{(}\PY{n}{df}\PY{o}{.}\PY{n}{POWER}\PY{p}{)}
         \PY{n}{trend} \PY{o}{=} \PY{n}{decomposition}\PY{o}{.}\PY{n}{trend}
         \PY{n}{seasonality} \PY{o}{=} \PY{n}{decomposition}\PY{o}{.}\PY{n}{seasonal}
         \PY{n}{residual} \PY{o}{=} \PY{n}{decomposition}\PY{o}{.}\PY{n}{resid}
         
         
         \PY{n}{fig}\PY{p}{,} \PY{n}{ax} \PY{o}{=} \PY{n}{plt}\PY{o}{.}\PY{n}{subplots}\PY{p}{(}\PY{n}{figsize} \PY{o}{=} \PY{p}{(}\PY{l+m+mi}{20}\PY{p}{,}\PY{l+m+mi}{2}\PY{p}{)}\PY{p}{)}
         \PY{n}{trend}\PY{p}{[}\PY{p}{:}\PY{l+m+mi}{200}\PY{p}{]}\PY{o}{.}\PY{n}{plot}\PY{p}{(}\PY{n}{ax}\PY{o}{=}\PY{n}{ax}\PY{p}{)}
         \PY{n}{fig}\PY{p}{,} \PY{n}{ax} \PY{o}{=} \PY{n}{plt}\PY{o}{.}\PY{n}{subplots}\PY{p}{(}\PY{l+m+mi}{1}\PY{p}{,}\PY{l+m+mi}{2}\PY{p}{,}\PY{n}{figsize} \PY{o}{=} \PY{p}{(}\PY{l+m+mi}{20}\PY{p}{,}\PY{l+m+mi}{2}\PY{p}{)}\PY{p}{)}
         \PY{n}{seasonality}\PY{p}{[}\PY{p}{:}\PY{l+m+mi}{200}\PY{p}{]}\PY{o}{.}\PY{n}{plot}\PY{p}{(}\PY{n}{ax}\PY{o}{=}\PY{n}{ax}\PY{p}{[}\PY{l+m+mi}{0}\PY{p}{]}\PY{p}{)}
         \PY{n}{residual}\PY{p}{[}\PY{p}{:}\PY{l+m+mi}{200}\PY{p}{]}\PY{o}{.}\PY{n}{plot}\PY{p}{(}\PY{n}{ax}\PY{o}{=}\PY{n}{ax}\PY{p}{[}\PY{l+m+mi}{1}\PY{p}{]}\PY{p}{)}
\end{Verbatim}


\begin{Verbatim}[commandchars=\\\{\}]
{\color{outcolor}Out[{\color{outcolor}16}]:} <matplotlib.axes.\_subplots.AxesSubplot at 0xc4a84f3710>
\end{Verbatim}
            
    \begin{center}
    \adjustimage{max size={0.9\linewidth}{0.9\paperheight}}{output_30_1.png}
    \end{center}
    { \hspace*{\fill} \\}
    
    \begin{center}
    \adjustimage{max size={0.9\linewidth}{0.9\paperheight}}{output_30_2.png}
    \end{center}
    { \hspace*{\fill} \\}
    
    \begin{center}
    \adjustimage{max size={0.9\linewidth}{0.9\paperheight}}{output_30_3.png}
    \end{center}
    { \hspace*{\fill} \\}
    
    \begin{center}
    \adjustimage{max size={0.9\linewidth}{0.9\paperheight}}{output_30_4.png}
    \end{center}
    { \hspace*{\fill} \\}
    
    \begin{center}
    \adjustimage{max size={0.9\linewidth}{0.9\paperheight}}{output_30_5.png}
    \end{center}
    { \hspace*{\fill} \\}
    
    \begin{center}
    \adjustimage{max size={0.9\linewidth}{0.9\paperheight}}{output_30_6.png}
    \end{center}
    { \hspace*{\fill} \\}
    
    Ebből látszik, hogy elég erős a napi szintű szezonalitása. Trendről sok
mindent nem lehet megállapítani, illetve főleg azt hogy nincs legalábbis
napi szinten nincs.

    \subsection{Autokorreláció}\label{autokorreluxe1ciuxf3}

    Ez után megnéztem, hogy milyen a korreláció a termelés mostani és jövő
beli értékei között.

    \begin{Verbatim}[commandchars=\\\{\}]
{\color{incolor}In [{\color{incolor}17}]:} \PY{n}{fig}\PY{p}{,} \PY{n}{ax} \PY{o}{=} \PY{n}{plt}\PY{o}{.}\PY{n}{subplots}\PY{p}{(}\PY{l+m+mi}{1}\PY{p}{,}\PY{l+m+mi}{2} \PY{p}{,}\PY{n}{figsize}\PY{o}{=}\PY{p}{(}\PY{l+m+mi}{20}\PY{p}{,}\PY{l+m+mi}{5}\PY{p}{)}\PY{p}{)}
         \PY{n}{lag\PYZus{}plot}\PY{p}{(}\PY{n}{df}\PY{p}{[}\PY{l+s+s1}{\PYZsq{}}\PY{l+s+s1}{POWER}\PY{l+s+s1}{\PYZsq{}}\PY{p}{]}\PY{o}{.}\PY{n}{tail}\PY{p}{(}\PY{l+m+mi}{250}\PY{p}{)}\PY{p}{,}\PY{n}{ax}\PY{o}{=}\PY{n}{ax}\PY{p}{[}\PY{l+m+mi}{0}\PY{p}{]}\PY{p}{)}
         \PY{n}{autocorrelation\PYZus{}plot}\PY{p}{(}\PY{n}{df}\PY{p}{[}\PY{l+s+s1}{\PYZsq{}}\PY{l+s+s1}{POWER}\PY{l+s+s1}{\PYZsq{}}\PY{p}{]}\PY{o}{.}\PY{n}{tail}\PY{p}{(}\PY{l+m+mi}{250}\PY{p}{)}\PY{p}{,}\PY{n}{ax}\PY{o}{=}\PY{n}{ax}\PY{p}{[}\PY{l+m+mi}{1}\PY{p}{]}\PY{p}{)}
\end{Verbatim}


\begin{Verbatim}[commandchars=\\\{\}]
{\color{outcolor}Out[{\color{outcolor}17}]:} <matplotlib.axes.\_subplots.AxesSubplot at 0xc4a962a2b0>
\end{Verbatim}
            
    \begin{center}
    \adjustimage{max size={0.9\linewidth}{0.9\paperheight}}{output_34_1.png}
    \end{center}
    { \hspace*{\fill} \\}
    
    Innen látszik, hogy van kap0csolat a kettő között. A második diagrammról
leolvasható, hogy kb. az 50.lag után kezd jelentéktelenebb lenni a
kapcsolat, ami 2 napot jelent.

    Ez után megnéztem, milyen a kapcsolat komponensenként.

    \begin{Verbatim}[commandchars=\\\{\}]
{\color{incolor}In [{\color{incolor}18}]:} \PY{n}{fig}\PY{p}{,} \PY{n}{ax} \PY{o}{=} \PY{n}{plt}\PY{o}{.}\PY{n}{subplots}\PY{p}{(}\PY{l+m+mi}{3}\PY{p}{,}\PY{l+m+mi}{2} \PY{p}{,}\PY{n}{figsize}\PY{o}{=}\PY{p}{(}\PY{l+m+mi}{20}\PY{p}{,}\PY{l+m+mi}{10}\PY{p}{)}\PY{p}{)}
         \PY{n}{lag\PYZus{}plot}\PY{p}{(}\PY{n}{trend}\PY{o}{.}\PY{n}{tail}\PY{p}{(}\PY{l+m+mi}{200}\PY{p}{)}\PY{p}{,}\PY{n}{ax}\PY{o}{=}\PY{n}{ax}\PY{p}{[}\PY{l+m+mi}{0}\PY{p}{]}\PY{p}{[}\PY{l+m+mi}{0}\PY{p}{]}\PY{p}{)}
         \PY{n}{autocorrelation\PYZus{}plot}\PY{p}{(}\PY{n}{trend}\PY{o}{.}\PY{n}{tail}\PY{p}{(}\PY{l+m+mi}{200}\PY{p}{)}\PY{p}{,}\PY{n}{ax}\PY{o}{=}\PY{n}{ax}\PY{p}{[}\PY{l+m+mi}{0}\PY{p}{]}\PY{p}{[}\PY{l+m+mi}{1}\PY{p}{]}\PY{p}{)}
         \PY{n}{lag\PYZus{}plot}\PY{p}{(}\PY{n}{seasonality}\PY{o}{.}\PY{n}{tail}\PY{p}{(}\PY{l+m+mi}{200}\PY{p}{)}\PY{p}{,}\PY{n}{ax}\PY{o}{=}\PY{n}{ax}\PY{p}{[}\PY{l+m+mi}{1}\PY{p}{]}\PY{p}{[}\PY{l+m+mi}{0}\PY{p}{]}\PY{p}{)}
         \PY{n}{autocorrelation\PYZus{}plot}\PY{p}{(}\PY{n}{seasonality}\PY{o}{.}\PY{n}{tail}\PY{p}{(}\PY{l+m+mi}{200}\PY{p}{)}\PY{p}{,}\PY{n}{ax}\PY{o}{=}\PY{n}{ax}\PY{p}{[}\PY{l+m+mi}{1}\PY{p}{]}\PY{p}{[}\PY{l+m+mi}{1}\PY{p}{]}\PY{p}{)}
         \PY{n}{lag\PYZus{}plot}\PY{p}{(}\PY{n}{residual}\PY{o}{.}\PY{n}{tail}\PY{p}{(}\PY{l+m+mi}{200}\PY{p}{)}\PY{p}{,}\PY{n}{ax}\PY{o}{=}\PY{n}{ax}\PY{p}{[}\PY{l+m+mi}{2}\PY{p}{]}\PY{p}{[}\PY{l+m+mi}{0}\PY{p}{]}\PY{p}{)}
         \PY{n}{autocorrelation\PYZus{}plot}\PY{p}{(}\PY{n}{residual}\PY{o}{.}\PY{n}{tail}\PY{p}{(}\PY{l+m+mi}{200}\PY{p}{)}\PY{p}{,}\PY{n}{ax}\PY{o}{=}\PY{n}{ax}\PY{p}{[}\PY{l+m+mi}{2}\PY{p}{]}\PY{p}{[}\PY{l+m+mi}{1}\PY{p}{]}\PY{p}{)}
\end{Verbatim}


\begin{Verbatim}[commandchars=\\\{\}]
{\color{outcolor}Out[{\color{outcolor}18}]:} <matplotlib.axes.\_subplots.AxesSubplot at 0xc4a82bda90>
\end{Verbatim}
            
    \begin{center}
    \adjustimage{max size={0.9\linewidth}{0.9\paperheight}}{output_37_1.png}
    \end{center}
    { \hspace*{\fill} \\}
    
    Ebből azt lehet megállapítani, hogy a szezonalitásból ered az erős
autokorreláció. Illetve a trendből is megállapítható egyfajta lineáris
kapcsolat.

    \begin{Verbatim}[commandchars=\\\{\}]
{\color{incolor}In [{\color{incolor}19}]:} \PY{n}{fig}\PY{p}{,} \PY{n}{ax} \PY{o}{=} \PY{n}{plt}\PY{o}{.}\PY{n}{subplots}\PY{p}{(}\PY{l+m+mi}{1}\PY{p}{,}\PY{l+m+mi}{2} \PY{p}{,}\PY{n}{figsize}\PY{o}{=}\PY{p}{(}\PY{l+m+mi}{20}\PY{p}{,}\PY{l+m+mi}{5}\PY{p}{)}\PY{p}{)}
         \PY{n}{lag\PYZus{}plot}\PY{p}{(}\PY{n}{df}\PY{p}{[}\PY{l+s+s1}{\PYZsq{}}\PY{l+s+s1}{POWER}\PY{l+s+s1}{\PYZsq{}}\PY{p}{]}\PY{o}{.}\PY{n}{tail}\PY{p}{(}\PY{l+m+mi}{250}\PY{p}{)}\PY{p}{,}\PY{n}{ax}\PY{o}{=}\PY{n}{ax}\PY{p}{[}\PY{l+m+mi}{0}\PY{p}{]}\PY{p}{)}
         \PY{n}{autocorrelation\PYZus{}plot}\PY{p}{(}\PY{n}{df}\PY{p}{[}\PY{l+s+s1}{\PYZsq{}}\PY{l+s+s1}{POWER}\PY{l+s+s1}{\PYZsq{}}\PY{p}{]}\PY{o}{.}\PY{n}{tail}\PY{p}{(}\PY{l+m+mi}{250}\PY{p}{)}\PY{p}{,}\PY{n}{ax}\PY{o}{=}\PY{n}{ax}\PY{p}{[}\PY{l+m+mi}{1}\PY{p}{]}\PY{p}{)}
\end{Verbatim}


\begin{Verbatim}[commandchars=\\\{\}]
{\color{outcolor}Out[{\color{outcolor}19}]:} <matplotlib.axes.\_subplots.AxesSubplot at 0xc4a84259b0>
\end{Verbatim}
            
    \begin{center}
    \adjustimage{max size={0.9\linewidth}{0.9\paperheight}}{output_39_1.png}
    \end{center}
    { \hspace*{\fill} \\}
    
    \subsection{Többi feature kapcsolata a
célváltozóval}\label{tuxf6bbi-feature-kapcsolata-a-cuxe9lvuxe1ltozuxf3val}

    Megnéztem egy két hetes időintervallumra, hogy hogy néznek ki az egyes
változók.

    \begin{Verbatim}[commandchars=\\\{\}]
{\color{incolor}In [{\color{incolor}20}]:} \PY{n}{df}\PY{p}{[}\PY{p}{:}\PY{n}{ONE\PYZus{}WEEK}\PY{o}{*}\PY{l+m+mi}{2}\PY{p}{]}\PY{o}{.}\PY{n}{plot}\PY{p}{(}\PY{n}{subplots}\PY{o}{=}\PY{k+kc}{True}\PY{p}{,}\PY{n}{figsize}\PY{o}{=}\PY{p}{(}\PY{l+m+mi}{20}\PY{p}{,}\PY{l+m+mi}{20}\PY{p}{)}\PY{p}{)}
\end{Verbatim}


\begin{Verbatim}[commandchars=\\\{\}]
{\color{outcolor}Out[{\color{outcolor}20}]:} array([<matplotlib.axes.\_subplots.AxesSubplot object at 0x000000C4A801B240>,
                <matplotlib.axes.\_subplots.AxesSubplot object at 0x000000C4A82FDA58>,
                <matplotlib.axes.\_subplots.AxesSubplot object at 0x000000C4A828A0F0>,
                <matplotlib.axes.\_subplots.AxesSubplot object at 0x000000C4A8288588>,
                <matplotlib.axes.\_subplots.AxesSubplot object at 0x000000C4A834FB00>,
                <matplotlib.axes.\_subplots.AxesSubplot object at 0x000000C4A834FB38>,
                <matplotlib.axes.\_subplots.AxesSubplot object at 0x000000C4A9804630>,
                <matplotlib.axes.\_subplots.AxesSubplot object at 0x000000C4A982BBA8>,
                <matplotlib.axes.\_subplots.AxesSubplot object at 0x000000C4A985E160>,
                <matplotlib.axes.\_subplots.AxesSubplot object at 0x000000C4A9A856D8>,
                <matplotlib.axes.\_subplots.AxesSubplot object at 0x000000C4A9AB1C50>,
                <matplotlib.axes.\_subplots.AxesSubplot object at 0x000000C4A9AE1208>,
                <matplotlib.axes.\_subplots.AxesSubplot object at 0x000000C4A9B0B780>],
               dtype=object)
\end{Verbatim}
            
    \begin{center}
    \adjustimage{max size={0.9\linewidth}{0.9\paperheight}}{output_42_1.png}
    \end{center}
    { \hspace*{\fill} \\}
    
    \subsubsection{Ami ebből első ránézésre
látszik:}\label{ami-ebbux151l-elsux151-ruxe1nuxe9zuxe9sre-luxe1tszik}

Ott voltak visszaesések a termelésben, ahol: * sokat esett az eső * nagy
volt a felhőzet Ami egész jelentéktelennek tűnik: * széljárás *
páratartalom Illetve látszik még, hogy az utolsó négy mező elég furcsán
változik, ez azért van, mert ezek akkumlált mezők.

    Szétbontottam az akkumlált mezőket.

    \begin{Verbatim}[commandchars=\\\{\}]
{\color{incolor}In [{\color{incolor}21}]:} \PY{n}{accumlated\PYZus{}columns} \PY{o}{=} \PY{p}{[}\PY{l+s+s2}{\PYZdq{}}\PY{l+s+s2}{SOLAR\PYZus{}RAD}\PY{l+s+s2}{\PYZdq{}}\PY{p}{,}\PY{l+s+s2}{\PYZdq{}}\PY{l+s+s2}{TERMAL\PYZus{}RAD}\PY{l+s+s2}{\PYZdq{}}\PY{p}{,}\PY{l+s+s2}{\PYZdq{}}\PY{l+s+s2}{TOP\PYZus{}NET\PYZus{}SOLAR\PYZus{}RAD}\PY{l+s+s2}{\PYZdq{}}\PY{p}{,}\PY{l+s+s2}{\PYZdq{}}\PY{l+s+s2}{TOTAL\PYZus{}PRECIPATION}\PY{l+s+s2}{\PYZdq{}}\PY{p}{]}
         \PY{n}{df} \PY{o}{=} \PY{n}{prepare\PYZus{}data}\PY{o}{.}\PY{n}{dissipate\PYZus{}features}\PY{p}{(}\PY{n}{df}\PY{o}{=}\PY{n}{df}\PY{p}{,} \PY{n}{columns}\PY{o}{=}\PY{n}{accumlated\PYZus{}columns}\PY{p}{)}
         \PY{n}{df}\PY{p}{[}\PY{p}{:}\PY{n}{ONE\PYZus{}WEEK}\PY{o}{*}\PY{l+m+mi}{2}\PY{p}{]}\PY{p}{[}\PY{n}{accumlated\PYZus{}columns}\PY{p}{]}\PY{o}{.}\PY{n}{plot}\PY{p}{(}\PY{n}{subplots}\PY{o}{=}\PY{k+kc}{True}\PY{p}{,}\PY{n}{figsize}\PY{o}{=}\PY{p}{(}\PY{l+m+mi}{20}\PY{p}{,}\PY{l+m+mi}{10}\PY{p}{)}\PY{p}{)}
\end{Verbatim}


\begin{Verbatim}[commandchars=\\\{\}]
{\color{outcolor}Out[{\color{outcolor}21}]:} array([<matplotlib.axes.\_subplots.AxesSubplot object at 0x000000C4A9E3D0B8>,
                <matplotlib.axes.\_subplots.AxesSubplot object at 0x000000C4AA9D65F8>,
                <matplotlib.axes.\_subplots.AxesSubplot object at 0x000000C4AA14AB70>,
                <matplotlib.axes.\_subplots.AxesSubplot object at 0x000000C4AA17E128>],
               dtype=object)
\end{Verbatim}
            
    \begin{center}
    \adjustimage{max size={0.9\linewidth}{0.9\paperheight}}{output_45_1.png}
    \end{center}
    { \hspace*{\fill} \\}
    
    \begin{Verbatim}[commandchars=\\\{\}]
{\color{incolor}In [{\color{incolor}22}]:} \PY{k+kn}{import} \PY{n+nn}{pandas} \PY{k}{as} \PY{n+nn}{pd}
         \PY{k+kn}{import} \PY{n+nn}{matplotlib}\PY{n+nn}{.}\PY{n+nn}{pyplot} \PY{k}{as} \PY{n+nn}{plt}
         
         \PY{k}{def} \PY{n+nf}{plot\PYZus{}scatters}\PY{p}{(}\PY{n}{df}\PY{p}{,} \PY{n}{target\PYZus{}column}\PY{p}{)}\PY{p}{:}
             \PY{n}{number\PYZus{}of\PYZus{}columns} \PY{o}{=} \PY{n+nb}{len}\PY{p}{(}\PY{n}{df}\PY{o}{.}\PY{n}{columns}\PY{o}{.}\PY{n}{values}\PY{p}{)}
             \PY{n}{plot\PYZus{}columns} \PY{o}{=} \PY{l+m+mi}{5}
             \PY{n}{whole\PYZus{}rows} \PY{o}{=} \PY{n+nb}{int}\PY{p}{(}\PY{n}{number\PYZus{}of\PYZus{}columns} \PY{o}{/} \PY{n}{plot\PYZus{}columns}\PY{p}{)}
             \PY{n}{extrarow} \PY{o}{=} \PY{l+m+mi}{1} \PY{k}{if} \PY{n}{number\PYZus{}of\PYZus{}columns} \PY{o}{\PYZpc{}} \PY{n}{plot\PYZus{}columns} \PY{o}{\PYZgt{}} \PY{l+m+mi}{0} \PY{k}{else} \PY{l+m+mi}{0}
         
             \PY{n}{row} \PY{o}{=} \PY{o}{\PYZhy{}}\PY{l+m+mi}{1}
             \PY{n}{column} \PY{o}{=} \PY{l+m+mi}{0}
             \PY{k}{for} \PY{n}{i}\PY{p}{,} \PY{n}{col} \PY{o+ow}{in} \PY{n+nb}{enumerate}\PY{p}{(}\PY{n}{df}\PY{o}{.}\PY{n}{columns}\PY{o}{.}\PY{n}{values}\PY{p}{)}\PY{p}{:}
         
                 \PY{n}{last\PYZus{}number\PYZus{}of\PYZus{}columns} \PY{o}{=} \PY{n}{number\PYZus{}of\PYZus{}columns}\PY{o}{\PYZpc{}}\PY{k}{plot\PYZus{}columns}
                 \PY{n}{is\PYZus{}last\PYZus{}row} \PY{o}{=} \PY{n+nb}{int}\PY{p}{(}\PY{n}{number\PYZus{}of\PYZus{}columns}\PY{o}{/}\PY{n}{plot\PYZus{}columns}\PY{p}{)}\PY{o}{*}\PY{n}{plot\PYZus{}columns} \PY{o}{\PYZlt{}}\PY{o}{=} \PY{n}{i}
         
                 \PY{k}{if}\PY{p}{(}\PY{n}{i}\PY{o}{\PYZpc{}}\PY{k}{plot\PYZus{}columns} == 0):
                     \PY{n}{row} \PY{o}{+}\PY{o}{=} \PY{l+m+mi}{1}
                     \PY{n}{column} \PY{o}{=} \PY{l+m+mi}{0}
                     \PY{k}{if} \PY{n}{is\PYZus{}last\PYZus{}row}\PY{p}{:}
                         \PY{n}{fig}\PY{p}{,} \PY{n}{ax} \PY{o}{=} \PY{n}{plt}\PY{o}{.}\PY{n}{subplots}\PY{p}{(}\PY{l+m+mi}{1}\PY{p}{,}\PY{n}{last\PYZus{}number\PYZus{}of\PYZus{}columns}\PY{p}{,}\PY{n}{figsize}\PY{o}{=}\PY{p}{(}\PY{l+m+mi}{20}\PY{p}{,}\PY{l+m+mi}{4}\PY{p}{)}\PY{p}{)}
                     \PY{k}{else}\PY{p}{:}
                         \PY{n}{fig}\PY{p}{,} \PY{n}{ax} \PY{o}{=} \PY{n}{plt}\PY{o}{.}\PY{n}{subplots}\PY{p}{(}\PY{l+m+mi}{1}\PY{p}{,}\PY{n}{plot\PYZus{}columns}\PY{p}{,}\PY{n}{figsize}\PY{o}{=}\PY{p}{(}\PY{l+m+mi}{20}\PY{p}{,}\PY{l+m+mi}{4}\PY{p}{)}\PY{p}{)}
                 \PY{k}{else}\PY{p}{:}
                     \PY{n}{column} \PY{o}{+}\PY{o}{=} \PY{l+m+mi}{1}
                 \PY{n}{df}\PY{o}{.}\PY{n}{plot}\PY{p}{(}\PY{n}{x}\PY{o}{=}\PY{p}{[}\PY{n}{col}\PY{p}{]}\PY{p}{,} \PY{n}{y}\PY{o}{=}\PY{p}{[}\PY{n}{target\PYZus{}column}\PY{p}{]}\PY{p}{,} \PY{n}{kind}\PY{o}{=}\PY{l+s+s2}{\PYZdq{}}\PY{l+s+s2}{scatter}\PY{l+s+s2}{\PYZdq{}}\PY{p}{,} \PY{n}{ax} \PY{o}{=} \PY{n}{ax}\PY{p}{[}\PY{n}{column}\PY{p}{]} \PY{p}{)}
\end{Verbatim}


    Majd megnéztem milyen kapcsolat van a célváltozó és az egyes változók
között.

    \begin{Verbatim}[commandchars=\\\{\}]
{\color{incolor}In [{\color{incolor}23}]:} \PY{n}{plot\PYZus{}scatters}\PY{p}{(}\PY{n}{df}\PY{p}{,} \PY{l+s+s2}{\PYZdq{}}\PY{l+s+s2}{POWER}\PY{l+s+s2}{\PYZdq{}}\PY{p}{)}
\end{Verbatim}


    \begin{center}
    \adjustimage{max size={0.9\linewidth}{0.9\paperheight}}{output_48_0.png}
    \end{center}
    { \hspace*{\fill} \\}
    
    \begin{center}
    \adjustimage{max size={0.9\linewidth}{0.9\paperheight}}{output_48_1.png}
    \end{center}
    { \hspace*{\fill} \\}
    
    \begin{center}
    \adjustimage{max size={0.9\linewidth}{0.9\paperheight}}{output_48_2.png}
    \end{center}
    { \hspace*{\fill} \\}
    
    \subsubsection{Amit innen meg tudok
állapítani:}\label{amit-innen-meg-tudok-uxe1llapuxedtani}

\begin{itemize}
\tightlist
\item
  Általában keveset esik az eső, de amikor esik, gyakoribbak az alacsony
  termelési értékek. Ehhez meg lehetne adni mondjuk egy kategorikus
  változót, ha egy adott értéknél többet esett az eső.
\item
  A előbbi megállapítás igaz a jég formájú esőre és a teljes
  csapadékmennyiségre is.
\item
  A napsugárzás mezőkben valószínüleg sok outlier van, mert ilyen magas
  napsugárzás értékekre 0 termelés elég elképzelhetetlen, de érdemes
  megnézni, hogy ez milyen időpontokban volt.
\end{itemize}

    Nézzük akkor a magas napsugárzás értékeknél mi történhetett.

    Milyen órákban nincs termelés általában

    \begin{Verbatim}[commandchars=\\\{\}]
{\color{incolor}In [{\color{incolor}24}]:} \PY{n}{df}\PY{p}{[}\PY{l+s+s2}{\PYZdq{}}\PY{l+s+s2}{HOUR}\PY{l+s+s2}{\PYZdq{}}\PY{p}{]} \PY{o}{=} \PY{n}{df}\PY{o}{.}\PY{n}{index}\PY{o}{.}\PY{n}{hour}
         \PY{n}{df}\PY{p}{[}\PY{l+s+s2}{\PYZdq{}}\PY{l+s+s2}{MONTH}\PY{l+s+s2}{\PYZdq{}}\PY{p}{]} \PY{o}{=} \PY{n}{df}\PY{o}{.}\PY{n}{index}\PY{o}{.}\PY{n}{month}
\end{Verbatim}


    \begin{Verbatim}[commandchars=\\\{\}]
{\color{incolor}In [{\color{incolor}25}]:} \PY{n}{df}\PY{p}{[}\PY{n}{df}\PY{p}{[}\PY{l+s+s2}{\PYZdq{}}\PY{l+s+s2}{POWER}\PY{l+s+s2}{\PYZdq{}}\PY{p}{]} \PY{o}{==} \PY{l+m+mi}{0}\PY{p}{]}\PY{o}{.}\PY{n}{HOUR}\PY{o}{.}\PY{n}{hist}\PY{p}{(}\PY{p}{)}
\end{Verbatim}


\begin{Verbatim}[commandchars=\\\{\}]
{\color{outcolor}Out[{\color{outcolor}25}]:} <matplotlib.axes.\_subplots.AxesSubplot at 0xc4aa69a710>
\end{Verbatim}
            
    \begin{center}
    \adjustimage{max size={0.9\linewidth}{0.9\paperheight}}{output_53_1.png}
    \end{center}
    { \hspace*{\fill} \\}
    
    9 és 21 óra között nincsen termelés leggyakrabban

    \begin{Verbatim}[commandchars=\\\{\}]
{\color{incolor}In [{\color{incolor}26}]:} \PY{n}{fig}\PY{p}{,} \PY{n}{ax} \PY{o}{=} \PY{n}{plt}\PY{o}{.}\PY{n}{subplots}\PY{p}{(}\PY{l+m+mi}{1}\PY{p}{,}\PY{l+m+mi}{2}\PY{p}{,}\PY{n}{figsize}\PY{o}{=}\PY{p}{(}\PY{l+m+mi}{20}\PY{p}{,}\PY{l+m+mi}{4}\PY{p}{)}\PY{p}{)}
         \PY{n}{df}\PY{o}{.}\PY{n}{SOLAR\PYZus{}RAD}\PY{p}{[}\PY{p}{:}\PY{n}{ONE\PYZus{}DAY}\PY{p}{]}\PY{o}{.}\PY{n}{plot}\PY{p}{(}\PY{n}{ax}\PY{o}{=}\PY{n}{ax}\PY{p}{[}\PY{l+m+mi}{0}\PY{p}{]}\PY{p}{)}
         \PY{n}{df}\PY{o}{.}\PY{n}{POWER}\PY{p}{[}\PY{p}{:}\PY{n}{ONE\PYZus{}DAY}\PY{p}{]}\PY{o}{.}\PY{n}{plot}\PY{p}{(}\PY{n}{ax}\PY{o}{=}\PY{n}{ax}\PY{p}{[}\PY{l+m+mi}{1}\PY{p}{]}\PY{p}{)}
\end{Verbatim}


\begin{Verbatim}[commandchars=\\\{\}]
{\color{outcolor}Out[{\color{outcolor}26}]:} <matplotlib.axes.\_subplots.AxesSubplot at 0xc4aa8e3588>
\end{Verbatim}
            
    \begin{center}
    \adjustimage{max size={0.9\linewidth}{0.9\paperheight}}{output_55_1.png}
    \end{center}
    { \hspace*{\fill} \\}
    
    Napsugázásra is igaz, hogy 9 és 21 óra között 0

    Nézzük, hogy ha nem termel a napelem, de a napsugárzás magas, az milyen
időpontokban van.

    \begin{Verbatim}[commandchars=\\\{\}]
{\color{incolor}In [{\color{incolor}27}]:} \PY{n}{mask} \PY{o}{=} \PY{p}{(}\PY{n}{df}\PY{p}{[}\PY{l+s+s2}{\PYZdq{}}\PY{l+s+s2}{SOLAR\PYZus{}RAD}\PY{l+s+s2}{\PYZdq{}}\PY{p}{]}\PY{o}{\PYZgt{}}\PY{l+m+mf}{0.5}\PY{o}{*}\PY{n+nb}{pow}\PY{p}{(}\PY{l+m+mi}{10}\PY{p}{,}\PY{l+m+mi}{7}\PY{p}{)}\PY{p}{)}  \PY{o}{\PYZam{}} \PY{p}{(}\PY{n}{df}\PY{o}{.}\PY{n}{POWER} \PY{o}{==} \PY{l+m+mi}{0}\PY{p}{)}
         \PY{n}{df\PYZus{}outlier} \PY{o}{=} \PY{n}{df}\PY{p}{[}\PY{n}{mask}\PY{p}{]}
         \PY{n}{df\PYZus{}outlier}\PY{p}{[}\PY{l+s+s2}{\PYZdq{}}\PY{l+s+s2}{HOUR}\PY{l+s+s2}{\PYZdq{}}\PY{p}{]} \PY{o}{=} \PY{n}{df\PYZus{}outlier}\PY{o}{.}\PY{n}{index}\PY{o}{.}\PY{n}{hour}
         \PY{n}{df\PYZus{}outlier}\PY{p}{[}\PY{l+s+s2}{\PYZdq{}}\PY{l+s+s2}{HOUR}\PY{l+s+s2}{\PYZdq{}}\PY{p}{]}\PY{o}{.}\PY{n}{hist}\PY{p}{(}\PY{p}{)}
\end{Verbatim}


\begin{Verbatim}[commandchars=\\\{\}]
{\color{outcolor}Out[{\color{outcolor}27}]:} <matplotlib.axes.\_subplots.AxesSubplot at 0xc4ab31a940>
\end{Verbatim}
            
    \begin{center}
    \adjustimage{max size={0.9\linewidth}{0.9\paperheight}}{output_58_1.png}
    \end{center}
    { \hspace*{\fill} \\}
    
    Ebből az látszik, hogy főleg 11 és 17 között jelenik meg, ami biztos nem
jó, mivel akkor van éjszaka.

    \begin{Verbatim}[commandchars=\\\{\}]
{\color{incolor}In [{\color{incolor}28}]:} \PY{n+nb}{print}\PY{p}{(}\PY{l+s+s2}{\PYZdq{}}\PY{l+s+si}{\PYZob{}0\PYZcb{}}\PY{l+s+s2}{ darab sorunk van összesen, ebből }\PY{l+s+si}{\PYZob{}1\PYZcb{}}\PY{l+s+s2}{ darab outlier.}\PY{l+s+s2}{\PYZdq{}}\PY{o}{.}\PY{n}{format}\PY{p}{(}\PY{n+nb}{len}\PY{p}{(}\PY{n}{df}\PY{p}{)}\PY{p}{,}\PY{n+nb}{len}\PY{p}{(}\PY{n}{df\PYZus{}outlier}\PY{p}{)}\PY{p}{)}\PY{p}{)}
\end{Verbatim}


    \begin{Verbatim}[commandchars=\\\{\}]
18984 darab sorunk van összesen, ebből 161 darab outlier.

    \end{Verbatim}

    Nézzük máshogy, nézzük

    \begin{Verbatim}[commandchars=\\\{\}]
{\color{incolor}In [{\color{incolor}29}]:} \PY{n}{mask} \PY{o}{=} \PY{p}{(}\PY{n}{df}\PY{p}{[}\PY{l+s+s2}{\PYZdq{}}\PY{l+s+s2}{HOUR}\PY{l+s+s2}{\PYZdq{}}\PY{p}{]}\PY{o}{\PYZgt{}}\PY{l+m+mi}{9}\PY{p}{)}  \PY{o}{\PYZam{}} \PY{p}{(}\PY{n}{df}\PY{o}{.}\PY{n}{HOUR} \PY{o}{\PYZlt{}} \PY{l+m+mi}{21}\PY{p}{)}
         \PY{c+c1}{\PYZsh{}df[mask][\PYZdq{}SOLAR\PYZus{}RAD\PYZdq{}].hist()}
         \PY{n+nb}{len}\PY{p}{(}\PY{n}{df}\PY{p}{[}\PY{n}{mask}\PY{p}{]}\PY{p}{)}
         \PY{n}{sns}\PY{o}{.}\PY{n}{kdeplot}\PY{p}{(}\PY{n}{df}\PY{p}{[}\PY{n}{mask}\PY{p}{]}\PY{p}{[}\PY{l+s+s2}{\PYZdq{}}\PY{l+s+s2}{SOLAR\PYZus{}RAD}\PY{l+s+s2}{\PYZdq{}}\PY{p}{]}\PY{p}{)}
\end{Verbatim}


\begin{Verbatim}[commandchars=\\\{\}]
{\color{outcolor}Out[{\color{outcolor}29}]:} <matplotlib.axes.\_subplots.AxesSubplot at 0xc4ab622d68>
\end{Verbatim}
            
    \begin{center}
    \adjustimage{max size={0.9\linewidth}{0.9\paperheight}}{output_62_1.png}
    \end{center}
    { \hspace*{\fill} \\}
    
    \begin{Verbatim}[commandchars=\\\{\}]
{\color{incolor}In [{\color{incolor}30}]:} \PY{n}{mask} \PY{o}{=} \PY{p}{(}\PY{n}{df}\PY{p}{[}\PY{l+s+s2}{\PYZdq{}}\PY{l+s+s2}{HOUR}\PY{l+s+s2}{\PYZdq{}}\PY{p}{]}\PY{o}{\PYZlt{}}\PY{l+m+mi}{9}\PY{p}{)}  \PY{o}{|} \PY{p}{(}\PY{n}{df}\PY{o}{.}\PY{n}{HOUR} \PY{o}{|} \PY{l+m+mi}{21}\PY{p}{)}
         \PY{n}{df}\PY{p}{[}\PY{n}{mask}\PY{p}{]}\PY{p}{[}\PY{l+s+s2}{\PYZdq{}}\PY{l+s+s2}{SOLAR\PYZus{}RAD}\PY{l+s+s2}{\PYZdq{}}\PY{p}{]}\PY{o}{.}\PY{n}{hist}\PY{p}{(}\PY{p}{)}
\end{Verbatim}


\begin{Verbatim}[commandchars=\\\{\}]
{\color{outcolor}Out[{\color{outcolor}30}]:} <matplotlib.axes.\_subplots.AxesSubplot at 0xc4ac655cc0>
\end{Verbatim}
            
    \begin{center}
    \adjustimage{max size={0.9\linewidth}{0.9\paperheight}}{output_63_1.png}
    \end{center}
    { \hspace*{\fill} \\}
    
    Ebből, látszik, hogy van pár irreálisan magas érték.

    \begin{Verbatim}[commandchars=\\\{\}]
{\color{incolor}In [{\color{incolor}31}]:} \PY{n}{mask} \PY{o}{=} \PY{p}{(}\PY{n}{df}\PY{p}{[}\PY{l+s+s2}{\PYZdq{}}\PY{l+s+s2}{HOUR}\PY{l+s+s2}{\PYZdq{}}\PY{p}{]}\PY{o}{\PYZgt{}}\PY{l+m+mi}{9}\PY{p}{)}  \PY{o}{\PYZam{}} \PY{p}{(}\PY{n}{df}\PY{o}{.}\PY{n}{HOUR} \PY{o}{\PYZlt{}} \PY{l+m+mi}{21}\PY{p}{)} \PY{o}{\PYZam{}} \PY{p}{(}\PY{n}{df}\PY{o}{.}\PY{n}{SOLAR\PYZus{}RAD} \PY{o}{\PYZgt{}} \PY{n}{df}\PY{o}{.}\PY{n}{SOLAR\PYZus{}RAD}\PY{o}{.}\PY{n}{median}\PY{p}{(}\PY{p}{)}\PY{p}{)}
         \PY{n+nb}{print}\PY{p}{(}\PY{l+s+s2}{\PYZdq{}}\PY{l+s+si}{\PYZob{}0\PYZcb{}}\PY{l+s+s2}{ darab sorunk van összesen, ebből }\PY{l+s+si}{\PYZob{}1\PYZcb{}}\PY{l+s+s2}{ darab outlier a solar rad tekintetében.}\PY{l+s+s2}{\PYZdq{}}\PY{o}{.}\PY{n}{format}\PY{p}{(}\PY{n+nb}{len}\PY{p}{(}\PY{n}{df}\PY{p}{)}\PY{p}{,}\PY{n+nb}{len}\PY{p}{(}\PY{n}{df}\PY{p}{[}\PY{n}{mask}\PY{p}{]}\PY{p}{)}\PY{p}{)}\PY{p}{)}
\end{Verbatim}


    \begin{Verbatim}[commandchars=\\\{\}]
18984 darab sorunk van összesen, ebből 419 darab outlier a solar rad tekintetében.

    \end{Verbatim}

    Felvettem a szezonalitás szerint szétbontott mezőket

    \begin{Verbatim}[commandchars=\\\{\}]
{\color{incolor}In [{\color{incolor}32}]:} \PY{n}{df}\PY{p}{[}\PY{l+s+s2}{\PYZdq{}}\PY{l+s+s2}{RESID}\PY{l+s+s2}{\PYZdq{}}\PY{p}{]} \PY{o}{=} \PY{n}{seasonal\PYZus{}decompose}\PY{p}{(}\PY{n}{df}\PY{o}{.}\PY{n}{POWER}\PY{p}{)}\PY{o}{.}\PY{n}{resid}
         \PY{n}{df}\PY{p}{[}\PY{l+s+s2}{\PYZdq{}}\PY{l+s+s2}{SEASONAL}\PY{l+s+s2}{\PYZdq{}}\PY{p}{]} \PY{o}{=} \PY{n}{seasonal\PYZus{}decompose}\PY{p}{(}\PY{n}{df}\PY{o}{.}\PY{n}{POWER}\PY{p}{)}\PY{o}{.}\PY{n}{seasonal}
         \PY{n}{df}\PY{p}{[}\PY{l+s+s2}{\PYZdq{}}\PY{l+s+s2}{TREND}\PY{l+s+s2}{\PYZdq{}}\PY{p}{]} \PY{o}{=} \PY{n}{seasonal\PYZus{}decompose}\PY{p}{(}\PY{n}{df}\PY{o}{.}\PY{n}{POWER}\PY{p}{)}\PY{o}{.}\PY{n}{trend}
\end{Verbatim}


    \begin{Verbatim}[commandchars=\\\{\}]
{\color{incolor}In [{\color{incolor}33}]:} \PY{n}{df}\PY{o}{.}\PY{n}{POWER}\PY{p}{[}\PY{p}{:}\PY{n}{ONE\PYZus{}WEEK}\PY{p}{]}\PY{o}{.}\PY{n}{plot}\PY{p}{(}\PY{n}{figsize}\PY{o}{=}\PY{p}{(}\PY{l+m+mi}{20}\PY{p}{,}\PY{l+m+mi}{5}\PY{p}{)}\PY{p}{)}
         \PY{n}{df}\PY{o}{.}\PY{n}{SEASONAL}\PY{p}{[}\PY{p}{:}\PY{n}{ONE\PYZus{}WEEK}\PY{p}{]}\PY{o}{.}\PY{n}{plot}\PY{p}{(}\PY{n}{figsize}\PY{o}{=}\PY{p}{(}\PY{l+m+mi}{20}\PY{p}{,}\PY{l+m+mi}{5}\PY{p}{)}\PY{p}{)}
         \PY{n}{df}\PY{o}{.}\PY{n}{TREND}\PY{p}{[}\PY{p}{:}\PY{n}{ONE\PYZus{}WEEK}\PY{p}{]}\PY{o}{.}\PY{n}{plot}\PY{p}{(}\PY{n}{figsize}\PY{o}{=}\PY{p}{(}\PY{l+m+mi}{20}\PY{p}{,}\PY{l+m+mi}{5}\PY{p}{)}\PY{p}{)}
         \PY{n}{df}\PY{o}{.}\PY{n}{RESID}\PY{p}{[}\PY{p}{:}\PY{n}{ONE\PYZus{}WEEK}\PY{p}{]}\PY{o}{.}\PY{n}{plot}\PY{p}{(}\PY{n}{figsize}\PY{o}{=}\PY{p}{(}\PY{l+m+mi}{20}\PY{p}{,}\PY{l+m+mi}{5}\PY{p}{)}\PY{p}{)}
\end{Verbatim}


\begin{Verbatim}[commandchars=\\\{\}]
{\color{outcolor}Out[{\color{outcolor}33}]:} <matplotlib.axes.\_subplots.AxesSubplot at 0xc4ac6917f0>
\end{Verbatim}
            
    \begin{center}
    \adjustimage{max size={0.9\linewidth}{0.9\paperheight}}{output_68_1.png}
    \end{center}
    { \hspace*{\fill} \\}
    
    \begin{Verbatim}[commandchars=\\\{\}]
{\color{incolor}In [{\color{incolor}34}]:} \PY{n}{ORIG} \PY{o}{=} \PY{n}{df}\PY{p}{[}\PY{l+s+s2}{\PYZdq{}}\PY{l+s+s2}{RESID}\PY{l+s+s2}{\PYZdq{}}\PY{p}{]}\PY{o}{+}\PY{n}{df}\PY{p}{[}\PY{l+s+s2}{\PYZdq{}}\PY{l+s+s2}{SEASONAL}\PY{l+s+s2}{\PYZdq{}}\PY{p}{]}\PY{o}{+}\PY{n}{df}\PY{p}{[}\PY{l+s+s2}{\PYZdq{}}\PY{l+s+s2}{TREND}\PY{l+s+s2}{\PYZdq{}}\PY{p}{]}
         \PY{n}{ORIG}\PY{p}{[}\PY{p}{:}\PY{n}{ONE\PYZus{}WEEK}\PY{p}{]}\PY{o}{.}\PY{n}{plot}\PY{p}{(}\PY{n}{figsize}\PY{o}{=}\PY{p}{(}\PY{l+m+mi}{20}\PY{p}{,}\PY{l+m+mi}{5}\PY{p}{)}\PY{p}{)}
         \PY{n}{df}\PY{o}{.}\PY{n}{POWER}\PY{p}{[}\PY{p}{:}\PY{n}{ONE\PYZus{}WEEK}\PY{p}{]}\PY{o}{.}\PY{n}{plot}\PY{p}{(}\PY{n}{figsize}\PY{o}{=}\PY{p}{(}\PY{l+m+mi}{20}\PY{p}{,}\PY{l+m+mi}{5}\PY{p}{)}\PY{p}{)}
\end{Verbatim}


\begin{Verbatim}[commandchars=\\\{\}]
{\color{outcolor}Out[{\color{outcolor}34}]:} <matplotlib.axes.\_subplots.AxesSubplot at 0xc4ac7676d8>
\end{Verbatim}
            
    \begin{center}
    \adjustimage{max size={0.9\linewidth}{0.9\paperheight}}{output_69_1.png}
    \end{center}
    { \hspace*{\fill} \\}
    
    Ebből biztosan látszik, hogy a resid, seasonal és trend mezők összege
visszaadja az eredeti függvényt.

    Megnéztem a korrelációt az összes mező között

    \begin{Verbatim}[commandchars=\\\{\}]
{\color{incolor}In [{\color{incolor}35}]:} \PY{n}{fig}\PY{p}{,} \PY{n}{ax} \PY{o}{=} \PY{n}{plt}\PY{o}{.}\PY{n}{subplots}\PY{p}{(}\PY{n}{figsize}\PY{o}{=}\PY{p}{(}\PY{l+m+mi}{14}\PY{p}{,}\PY{l+m+mi}{8}\PY{p}{)}\PY{p}{)}         \PY{c+c1}{\PYZsh{} Sample figsize in inches}
         
         \PY{n}{f} \PY{o}{=} \PY{p}{(}
             \PY{n}{df}\PY{o}{.}\PY{n}{loc}\PY{p}{[}\PY{p}{:}\PY{p}{,} \PY{n}{df}\PY{o}{.}\PY{n}{columns}\PY{p}{]}
                 
         \PY{p}{)}\PY{o}{.}\PY{n}{corr}\PY{p}{(}\PY{p}{)}
         
         \PY{n}{sns}\PY{o}{.}\PY{n}{heatmap}\PY{p}{(}\PY{n}{f}\PY{p}{,} \PY{n}{annot}\PY{o}{=}\PY{k+kc}{True}\PY{p}{,} \PY{n}{ax}\PY{o}{=}\PY{n}{ax}\PY{p}{)}
\end{Verbatim}


\begin{Verbatim}[commandchars=\\\{\}]
{\color{outcolor}Out[{\color{outcolor}35}]:} <matplotlib.axes.\_subplots.AxesSubplot at 0xc4ac7d7eb8>
\end{Verbatim}
            
    \begin{center}
    \adjustimage{max size={0.9\linewidth}{0.9\paperheight}}{output_72_1.png}
    \end{center}
    { \hspace*{\fill} \\}
    
    Innen látszik, hogy az összes eredeti feature közül legjobban a relatív
páratartalom, a hőmérsékelet és a napsugárzás korrelálnak a
célválztozóval.

Az általam létrehozottak közül pedig a szezonalitás.

Az ötletem az lenne hogy a szezonalitást függvényként tárolom majd, így
minden időponthoz kiszámolható. Gépi tanulással pedig a residual és
talán a trend komponenseket predictelem.

    \begin{Verbatim}[commandchars=\\\{\}]
{\color{incolor}In [{\color{incolor}36}]:} \PY{n}{df}\PY{p}{[}\PY{l+s+s2}{\PYZdq{}}\PY{l+s+s2}{RT}\PY{l+s+s2}{\PYZdq{}}\PY{p}{]} \PY{o}{=} \PY{n}{df}\PY{p}{[}\PY{l+s+s2}{\PYZdq{}}\PY{l+s+s2}{SEASONAL}\PY{l+s+s2}{\PYZdq{}}\PY{p}{]} \PY{o}{+} \PY{n}{df}\PY{p}{[}\PY{l+s+s2}{\PYZdq{}}\PY{l+s+s2}{RESID}\PY{l+s+s2}{\PYZdq{}}\PY{p}{]}
\end{Verbatim}


    \begin{Verbatim}[commandchars=\\\{\}]
{\color{incolor}In [{\color{incolor}37}]:} \PY{n}{df\PYZus{}to\PYZus{}plot} \PY{o}{=} \PY{n}{df}\PY{o}{.}\PY{n}{drop}\PY{p}{(}\PY{p}{[}\PY{l+s+s2}{\PYZdq{}}\PY{l+s+s2}{SEASONAL}\PY{l+s+s2}{\PYZdq{}}\PY{p}{,}\PY{l+s+s2}{\PYZdq{}}\PY{l+s+s2}{HOUR}\PY{l+s+s2}{\PYZdq{}}\PY{p}{,}\PY{l+s+s2}{\PYZdq{}}\PY{l+s+s2}{MONTH}\PY{l+s+s2}{\PYZdq{}}\PY{p}{,}\PY{l+s+s2}{\PYZdq{}}\PY{l+s+s2}{RESID}\PY{l+s+s2}{\PYZdq{}}\PY{p}{,}\PY{l+s+s2}{\PYZdq{}}\PY{l+s+s2}{TREND}\PY{l+s+s2}{\PYZdq{}}\PY{p}{]}\PY{p}{,}\PY{n}{axis}\PY{o}{=}\PY{l+m+mi}{1}\PY{p}{)}
         \PY{n}{fig}\PY{p}{,} \PY{n}{ax} \PY{o}{=} \PY{n}{plt}\PY{o}{.}\PY{n}{subplots}\PY{p}{(}\PY{n}{figsize}\PY{o}{=}\PY{p}{(}\PY{l+m+mi}{14}\PY{p}{,}\PY{l+m+mi}{8}\PY{p}{)}\PY{p}{)}         \PY{c+c1}{\PYZsh{} Sample figsize in inches}
         
         \PY{n}{f} \PY{o}{=} \PY{p}{(}
             \PY{n}{df\PYZus{}to\PYZus{}plot}\PY{o}{.}\PY{n}{loc}\PY{p}{[}\PY{p}{:}\PY{p}{,} \PY{n}{df\PYZus{}to\PYZus{}plot}\PY{o}{.}\PY{n}{columns}\PY{p}{]}
                 
         \PY{p}{)}\PY{o}{.}\PY{n}{corr}\PY{p}{(}\PY{p}{)}
         
         \PY{n}{sns}\PY{o}{.}\PY{n}{heatmap}\PY{p}{(}\PY{n}{f}\PY{p}{,} \PY{n}{annot}\PY{o}{=}\PY{k+kc}{True}\PY{p}{,} \PY{n}{ax}\PY{o}{=}\PY{n}{ax}\PY{p}{)}
\end{Verbatim}


\begin{Verbatim}[commandchars=\\\{\}]
{\color{outcolor}Out[{\color{outcolor}37}]:} <matplotlib.axes.\_subplots.AxesSubplot at 0xc4aa95f630>
\end{Verbatim}
            
    \begin{center}
    \adjustimage{max size={0.9\linewidth}{0.9\paperheight}}{output_75_1.png}
    \end{center}
    { \hspace*{\fill} \\}
    
    Relatív páratartalom, napsugárzás és hőmérséklet amik fontosabbak.

    \begin{Verbatim}[commandchars=\\\{\}]
{\color{incolor}In [{\color{incolor}38}]:} \PY{n}{plot\PYZus{}scatters}\PY{p}{(}\PY{n}{df}\PY{o}{.}\PY{n}{sample}\PY{p}{(}\PY{l+m+mi}{200}\PY{p}{)}\PY{p}{,}\PY{l+s+s2}{\PYZdq{}}\PY{l+s+s2}{RT}\PY{l+s+s2}{\PYZdq{}}\PY{p}{)}
\end{Verbatim}


    \begin{center}
    \adjustimage{max size={0.9\linewidth}{0.9\paperheight}}{output_77_0.png}
    \end{center}
    { \hspace*{\fill} \\}
    
    \begin{center}
    \adjustimage{max size={0.9\linewidth}{0.9\paperheight}}{output_77_1.png}
    \end{center}
    { \hspace*{\fill} \\}
    
    \begin{center}
    \adjustimage{max size={0.9\linewidth}{0.9\paperheight}}{output_77_2.png}
    \end{center}
    { \hspace*{\fill} \\}
    
    \begin{center}
    \adjustimage{max size={0.9\linewidth}{0.9\paperheight}}{output_77_3.png}
    \end{center}
    { \hspace*{\fill} \\}
    
    \begin{Verbatim}[commandchars=\\\{\}]
{\color{incolor}In [{\color{incolor}39}]:} \PY{k+kn}{import} \PY{n+nn}{pandas} \PY{k}{as} \PY{n+nn}{pd}
         \PY{k+kn}{import} \PY{n+nn}{matplotlib}\PY{n+nn}{.}\PY{n+nn}{pyplot} \PY{k}{as} \PY{n+nn}{plt}
         
         \PY{k}{def} \PY{n+nf}{plot\PYZus{}lags\PYZus{}and\PYZus{}auto}\PY{p}{(}\PY{n}{df}\PY{p}{)}\PY{p}{:}
             \PY{n}{columns} \PY{o}{=} \PY{n+nb}{len}\PY{p}{(}\PY{n}{df}\PY{o}{.}\PY{n}{columns}\PY{p}{)}
             \PY{k}{for} \PY{n}{col} \PY{o+ow}{in} \PY{n}{df}\PY{o}{.}\PY{n}{columns}\PY{p}{:}
                 \PY{n}{fig}\PY{p}{,} \PY{n}{ax} \PY{o}{=} \PY{n}{plt}\PY{o}{.}\PY{n}{subplots}\PY{p}{(}\PY{l+m+mi}{1}\PY{p}{,}\PY{l+m+mi}{2}\PY{p}{,}\PY{n}{figsize}\PY{o}{=}\PY{p}{(}\PY{l+m+mi}{20}\PY{p}{,}\PY{l+m+mi}{5}\PY{p}{)}\PY{p}{)}
                 \PY{n}{fig}\PY{o}{.}\PY{n}{suptitle}\PY{p}{(}\PY{n}{col}\PY{p}{,} \PY{n}{fontsize}\PY{o}{=}\PY{l+m+mi}{16}\PY{p}{)}
                 \PY{n}{lag\PYZus{}plot}\PY{p}{(}\PY{n}{df}\PY{p}{[}\PY{n}{col}\PY{p}{]}\PY{o}{.}\PY{n}{tail}\PY{p}{(}\PY{l+m+mi}{250}\PY{p}{)}\PY{p}{,}\PY{n}{ax} \PY{o}{=}\PY{n}{ax}\PY{p}{[}\PY{l+m+mi}{0}\PY{p}{]}\PY{p}{)}
                 \PY{n}{autocorrelation\PYZus{}plot}\PY{p}{(}\PY{n}{df}\PY{p}{[}\PY{n}{col}\PY{p}{]}\PY{o}{.}\PY{n}{tail}\PY{p}{(}\PY{l+m+mi}{250}\PY{p}{)}\PY{p}{,} \PY{n}{ax} \PY{o}{=} \PY{n}{ax}\PY{p}{[}\PY{l+m+mi}{1}\PY{p}{]}\PY{p}{)}
\end{Verbatim}


    \begin{Verbatim}[commandchars=\\\{\}]
{\color{incolor}In [{\color{incolor}40}]:} \PY{n}{plot\PYZus{}lags\PYZus{}and\PYZus{}auto}\PY{p}{(}\PY{n}{df}\PY{p}{)}
\end{Verbatim}


    \begin{center}
    \adjustimage{max size={0.9\linewidth}{0.9\paperheight}}{output_79_0.png}
    \end{center}
    { \hspace*{\fill} \\}
    
    \begin{center}
    \adjustimage{max size={0.9\linewidth}{0.9\paperheight}}{output_79_1.png}
    \end{center}
    { \hspace*{\fill} \\}
    
    \begin{center}
    \adjustimage{max size={0.9\linewidth}{0.9\paperheight}}{output_79_2.png}
    \end{center}
    { \hspace*{\fill} \\}
    
    \begin{center}
    \adjustimage{max size={0.9\linewidth}{0.9\paperheight}}{output_79_3.png}
    \end{center}
    { \hspace*{\fill} \\}
    
    \begin{center}
    \adjustimage{max size={0.9\linewidth}{0.9\paperheight}}{output_79_4.png}
    \end{center}
    { \hspace*{\fill} \\}
    
    \begin{center}
    \adjustimage{max size={0.9\linewidth}{0.9\paperheight}}{output_79_5.png}
    \end{center}
    { \hspace*{\fill} \\}
    
    \begin{center}
    \adjustimage{max size={0.9\linewidth}{0.9\paperheight}}{output_79_6.png}
    \end{center}
    { \hspace*{\fill} \\}
    
    \begin{center}
    \adjustimage{max size={0.9\linewidth}{0.9\paperheight}}{output_79_7.png}
    \end{center}
    { \hspace*{\fill} \\}
    
    \begin{center}
    \adjustimage{max size={0.9\linewidth}{0.9\paperheight}}{output_79_8.png}
    \end{center}
    { \hspace*{\fill} \\}
    
    \begin{center}
    \adjustimage{max size={0.9\linewidth}{0.9\paperheight}}{output_79_9.png}
    \end{center}
    { \hspace*{\fill} \\}
    
    \begin{center}
    \adjustimage{max size={0.9\linewidth}{0.9\paperheight}}{output_79_10.png}
    \end{center}
    { \hspace*{\fill} \\}
    
    \begin{center}
    \adjustimage{max size={0.9\linewidth}{0.9\paperheight}}{output_79_11.png}
    \end{center}
    { \hspace*{\fill} \\}
    
    \begin{center}
    \adjustimage{max size={0.9\linewidth}{0.9\paperheight}}{output_79_12.png}
    \end{center}
    { \hspace*{\fill} \\}
    
    \begin{center}
    \adjustimage{max size={0.9\linewidth}{0.9\paperheight}}{output_79_13.png}
    \end{center}
    { \hspace*{\fill} \\}
    
    \begin{center}
    \adjustimage{max size={0.9\linewidth}{0.9\paperheight}}{output_79_14.png}
    \end{center}
    { \hspace*{\fill} \\}
    
    \begin{center}
    \adjustimage{max size={0.9\linewidth}{0.9\paperheight}}{output_79_15.png}
    \end{center}
    { \hspace*{\fill} \\}
    
    \begin{center}
    \adjustimage{max size={0.9\linewidth}{0.9\paperheight}}{output_79_16.png}
    \end{center}
    { \hspace*{\fill} \\}
    
    \begin{center}
    \adjustimage{max size={0.9\linewidth}{0.9\paperheight}}{output_79_17.png}
    \end{center}
    { \hspace*{\fill} \\}
    
    \begin{center}
    \adjustimage{max size={0.9\linewidth}{0.9\paperheight}}{output_79_18.png}
    \end{center}
    { \hspace*{\fill} \\}
    
    Innen azok a mezők használhatóak csak, amiknek 24 lag után is viszonylag
erős autokorrelációjuk van. Ezek pedig: * Surface pressure * Relative
humidity * WIND U * WIND V * Temperature

    Ezek alapján amiket fel fogok használni featureként azok a surface
pressure és a temperature. Illetve a célváltozót is.

    \begin{Verbatim}[commandchars=\\\{\}]
{\color{incolor}In [{\color{incolor}41}]:} \PY{n}{df\PYZus{}to\PYZus{}plot} \PY{o}{=} \PY{n}{df}\PY{o}{.}\PY{n}{drop}\PY{p}{(}\PY{p}{[}\PY{l+s+s2}{\PYZdq{}}\PY{l+s+s2}{SEASONAL}\PY{l+s+s2}{\PYZdq{}}\PY{p}{,}\PY{l+s+s2}{\PYZdq{}}\PY{l+s+s2}{HOUR}\PY{l+s+s2}{\PYZdq{}}\PY{p}{,}\PY{l+s+s2}{\PYZdq{}}\PY{l+s+s2}{MONTH}\PY{l+s+s2}{\PYZdq{}}\PY{p}{,}\PY{l+s+s2}{\PYZdq{}}\PY{l+s+s2}{RESID}\PY{l+s+s2}{\PYZdq{}}\PY{p}{,}\PY{l+s+s2}{\PYZdq{}}\PY{l+s+s2}{TREND}\PY{l+s+s2}{\PYZdq{}}\PY{p}{]}\PY{p}{,}\PY{n}{axis}\PY{o}{=}\PY{l+m+mi}{1}\PY{p}{)}\PY{o}{.}\PY{n}{resample}\PY{p}{(}\PY{l+s+s1}{\PYZsq{}}\PY{l+s+s1}{D}\PY{l+s+s1}{\PYZsq{}}\PY{p}{)}\PY{o}{.}\PY{n}{mean}\PY{p}{(}\PY{p}{)}
         \PY{n}{fig}\PY{p}{,} \PY{n}{ax} \PY{o}{=} \PY{n}{plt}\PY{o}{.}\PY{n}{subplots}\PY{p}{(}\PY{n}{figsize}\PY{o}{=}\PY{p}{(}\PY{l+m+mi}{14}\PY{p}{,}\PY{l+m+mi}{8}\PY{p}{)}\PY{p}{)}         \PY{c+c1}{\PYZsh{} Sample figsize in inches}
         
         \PY{n}{f} \PY{o}{=} \PY{p}{(}
             \PY{n}{df\PYZus{}to\PYZus{}plot}\PY{o}{.}\PY{n}{loc}\PY{p}{[}\PY{p}{:}\PY{p}{,} \PY{n}{df\PYZus{}to\PYZus{}plot}\PY{o}{.}\PY{n}{columns}\PY{p}{]}
                 
         \PY{p}{)}\PY{o}{.}\PY{n}{corr}\PY{p}{(}\PY{p}{)}
         
         \PY{n}{sns}\PY{o}{.}\PY{n}{heatmap}\PY{p}{(}\PY{n}{f}\PY{p}{,} \PY{n}{annot}\PY{o}{=}\PY{k+kc}{True}\PY{p}{,} \PY{n}{ax}\PY{o}{=}\PY{n}{ax}\PY{p}{)}
\end{Verbatim}


\begin{Verbatim}[commandchars=\\\{\}]
{\color{outcolor}Out[{\color{outcolor}41}]:} <matplotlib.axes.\_subplots.AxesSubplot at 0xc4ae8b9e10>
\end{Verbatim}
            
    \begin{center}
    \adjustimage{max size={0.9\linewidth}{0.9\paperheight}}{output_82_1.png}
    \end{center}
    { \hspace*{\fill} \\}
    
    Később talán az esőt is beleveszem, illetve a felhőtakarót.

    Rolling apply segítségével létrehozom a mozgó átlagot 24 órás ablakra a
fennt említett mezőkhöz.

    \begin{Verbatim}[commandchars=\\\{\}]
{\color{incolor}In [{\color{incolor}42}]:} \PY{n}{df\PYZus{}to\PYZus{}train} \PY{o}{=} \PY{n}{df}\PY{p}{[}\PY{p}{[}\PY{l+s+s2}{\PYZdq{}}\PY{l+s+s2}{POWER}\PY{l+s+s2}{\PYZdq{}}\PY{p}{,}\PY{l+s+s2}{\PYZdq{}}\PY{l+s+s2}{SURFACE\PYZus{}PRESSURE}\PY{l+s+s2}{\PYZdq{}}\PY{p}{,}\PY{l+s+s2}{\PYZdq{}}\PY{l+s+s2}{TEMPERATURE}\PY{l+s+s2}{\PYZdq{}}\PY{p}{,}\PY{l+s+s2}{\PYZdq{}}\PY{l+s+s2}{HOUR}\PY{l+s+s2}{\PYZdq{}}\PY{p}{,}\PY{l+s+s2}{\PYZdq{}}\PY{l+s+s2}{MONTH}\PY{l+s+s2}{\PYZdq{}}\PY{p}{]}\PY{p}{]}
         \PY{k}{for} \PY{n}{column} \PY{o+ow}{in} \PY{p}{[}\PY{l+s+s2}{\PYZdq{}}\PY{l+s+s2}{POWER}\PY{l+s+s2}{\PYZdq{}}\PY{p}{,}\PY{l+s+s2}{\PYZdq{}}\PY{l+s+s2}{SURFACE\PYZus{}PRESSURE}\PY{l+s+s2}{\PYZdq{}}\PY{p}{,}\PY{l+s+s2}{\PYZdq{}}\PY{l+s+s2}{TEMPERATURE}\PY{l+s+s2}{\PYZdq{}}\PY{p}{]}\PY{p}{:}
             \PY{n}{rolling\PYZus{}column} \PY{o}{=} \PY{n}{df\PYZus{}to\PYZus{}train}\PY{p}{[}\PY{n}{column}\PY{p}{]}\PY{o}{.}\PY{n}{rolling}\PY{p}{(}\PY{n}{window} \PY{o}{=} \PY{l+m+mi}{24}\PY{p}{)}
             \PY{n}{shift} \PY{o}{=} \PY{l+m+mi}{24}
             \PY{n}{df\PYZus{}to\PYZus{}train}\PY{p}{[}\PY{l+s+s2}{\PYZdq{}}\PY{l+s+s2}{ROLLING\PYZus{}MEAN\PYZus{}}\PY{l+s+s2}{\PYZdq{}}\PY{o}{+}\PY{n}{column}\PY{p}{]} \PY{o}{=} \PY{n}{rolling\PYZus{}column}\PY{o}{.}\PY{n}{mean}\PY{p}{(}\PY{p}{)}\PY{o}{.}\PY{n}{shift}\PY{p}{(}\PY{n}{shift}\PY{p}{)}
             \PY{n}{df\PYZus{}to\PYZus{}train}\PY{p}{[}\PY{l+s+s2}{\PYZdq{}}\PY{l+s+s2}{ROLLING\PYZus{}MAX\PYZus{}}\PY{l+s+s2}{\PYZdq{}}\PY{o}{+}\PY{n}{column}\PY{p}{]} \PY{o}{=} \PY{n}{rolling\PYZus{}column}\PY{o}{.}\PY{n}{max}\PY{p}{(}\PY{p}{)}\PY{o}{.}\PY{n}{shift}\PY{p}{(}\PY{n}{shift}\PY{p}{)}
             \PY{n}{df\PYZus{}to\PYZus{}train}\PY{p}{[}\PY{l+s+s2}{\PYZdq{}}\PY{l+s+s2}{ROLLING\PYZus{}SUM\PYZus{}}\PY{l+s+s2}{\PYZdq{}}\PY{o}{+}\PY{n}{column}\PY{p}{]} \PY{o}{=} \PY{n}{rolling\PYZus{}column}\PY{o}{.}\PY{n}{sum}\PY{p}{(}\PY{p}{)}\PY{o}{.}\PY{n}{shift}\PY{p}{(}\PY{n}{shift}\PY{p}{)}
             \PY{n}{df\PYZus{}to\PYZus{}train}\PY{p}{[}\PY{l+s+s2}{\PYZdq{}}\PY{l+s+s2}{ROLLING\PYZus{}MEDIAN\PYZus{}}\PY{l+s+s2}{\PYZdq{}}\PY{o}{+}\PY{n}{column}\PY{p}{]} \PY{o}{=} \PY{n}{rolling\PYZus{}column}\PY{o}{.}\PY{n}{median}\PY{p}{(}\PY{p}{)}\PY{o}{.}\PY{n}{shift}\PY{p}{(}\PY{n}{shift}\PY{p}{)}
             \PY{n}{df\PYZus{}to\PYZus{}train}\PY{p}{[}\PY{l+s+s2}{\PYZdq{}}\PY{l+s+s2}{ROLLING\PYZus{}STD\PYZus{}}\PY{l+s+s2}{\PYZdq{}}\PY{o}{+}\PY{n}{column}\PY{p}{]} \PY{o}{=} \PY{n}{rolling\PYZus{}column}\PY{o}{.}\PY{n}{std}\PY{p}{(}\PY{p}{)}\PY{o}{.}\PY{n}{shift}\PY{p}{(}\PY{n}{shift}\PY{p}{)}
             \PY{n}{df\PYZus{}to\PYZus{}train}\PY{p}{[}\PY{l+s+s2}{\PYZdq{}}\PY{l+s+s2}{ROLLING\PYZus{}VAR\PYZus{}}\PY{l+s+s2}{\PYZdq{}}\PY{o}{+}\PY{n}{column}\PY{p}{]} \PY{o}{=} \PY{n}{rolling\PYZus{}column}\PY{o}{.}\PY{n}{var}\PY{p}{(}\PY{p}{)}\PY{o}{.}\PY{n}{shift}\PY{p}{(}\PY{n}{shift}\PY{p}{)}
         
         \PY{n}{X\PYZus{}train} \PY{o}{=} \PY{n}{df\PYZus{}to\PYZus{}train}\PY{p}{[}\PY{p}{:}\PY{n}{ONE\PYZus{}YEAR}\PY{p}{]}\PY{o}{.}\PY{n}{dropna}\PY{p}{(}\PY{n}{axis} \PY{o}{=} \PY{l+m+mi}{0}\PY{p}{)}\PY{o}{.}\PY{n}{drop}\PY{p}{(}\PY{p}{[}\PY{l+s+s2}{\PYZdq{}}\PY{l+s+s2}{POWER}\PY{l+s+s2}{\PYZdq{}}\PY{p}{,}\PY{l+s+s2}{\PYZdq{}}\PY{l+s+s2}{SURFACE\PYZus{}PRESSURE}\PY{l+s+s2}{\PYZdq{}}\PY{p}{,}\PY{l+s+s2}{\PYZdq{}}\PY{l+s+s2}{TEMPERATURE}\PY{l+s+s2}{\PYZdq{}}\PY{p}{]}\PY{p}{,}\PY{n}{axis}\PY{o}{=}\PY{l+m+mi}{1}\PY{p}{)}
         \PY{n}{y\PYZus{}train} \PY{o}{=} \PY{n}{df\PYZus{}to\PYZus{}train}\PY{p}{[}\PY{p}{:}\PY{n}{ONE\PYZus{}YEAR}\PY{p}{]}\PY{o}{.}\PY{n}{dropna}\PY{p}{(}\PY{n}{axis} \PY{o}{=} \PY{l+m+mi}{0}\PY{p}{)}\PY{o}{.}\PY{n}{POWER}
         \PY{n}{X\PYZus{}test} \PY{o}{=} \PY{n}{df\PYZus{}to\PYZus{}train}\PY{p}{[}\PY{n}{ONE\PYZus{}YEAR}\PY{p}{:}\PY{p}{]}\PY{o}{.}\PY{n}{dropna}\PY{p}{(}\PY{n}{axis} \PY{o}{=} \PY{l+m+mi}{0}\PY{p}{)}\PY{o}{.}\PY{n}{drop}\PY{p}{(}\PY{p}{[}\PY{l+s+s2}{\PYZdq{}}\PY{l+s+s2}{POWER}\PY{l+s+s2}{\PYZdq{}}\PY{p}{,}\PY{l+s+s2}{\PYZdq{}}\PY{l+s+s2}{SURFACE\PYZus{}PRESSURE}\PY{l+s+s2}{\PYZdq{}}\PY{p}{,}\PY{l+s+s2}{\PYZdq{}}\PY{l+s+s2}{TEMPERATURE}\PY{l+s+s2}{\PYZdq{}}\PY{p}{]}\PY{p}{,}\PY{n}{axis}\PY{o}{=}\PY{l+m+mi}{1}\PY{p}{)}
         \PY{n}{y\PYZus{}test} \PY{o}{=} \PY{n}{df\PYZus{}to\PYZus{}train}\PY{p}{[}\PY{n}{ONE\PYZus{}YEAR}\PY{p}{:}\PY{p}{]}\PY{o}{.}\PY{n}{dropna}\PY{p}{(}\PY{n}{axis} \PY{o}{=} \PY{l+m+mi}{0}\PY{p}{)}\PY{o}{.}\PY{n}{POWER}
         \PY{c+c1}{\PYZsh{}print(len(df\PYZus{}to\PYZus{}train))}
         \PY{c+c1}{\PYZsh{}print(len(df\PYZus{}to\PYZus{}train.dropna(axis=0)))}
\end{Verbatim}


    Először, a power-re predictelek, mert a szezonalitást még nem tudom hogy
tároljam le.

    \begin{Verbatim}[commandchars=\\\{\}]
{\color{incolor}In [{\color{incolor}43}]:} \PY{k+kn}{from} \PY{n+nn}{sklearn}\PY{n+nn}{.}\PY{n+nn}{metrics} \PY{k}{import} \PY{n}{mean\PYZus{}squared\PYZus{}error}\PY{p}{,}\PY{n}{explained\PYZus{}variance\PYZus{}score}
         \PY{k}{def} \PY{n+nf}{train\PYZus{}predict\PYZus{}and\PYZus{}score}\PY{p}{(}\PY{n}{X\PYZus{}train}\PY{p}{,} \PY{n}{y\PYZus{}train}\PY{p}{,}\PY{n}{X\PYZus{}test}\PY{p}{,}\PY{n}{y\PYZus{}test}\PY{p}{,}\PY{n}{models}\PY{p}{,}\PY{n}{names}\PY{p}{)}\PY{p}{:}
             \PY{k}{for} \PY{n}{name}\PY{p}{,}\PY{n}{model} \PY{o+ow}{in} \PY{n+nb}{zip}\PY{p}{(}\PY{n}{names}\PY{p}{,}\PY{n}{models}\PY{p}{)}\PY{p}{:}
                 \PY{n}{model}\PY{o}{.}\PY{n}{fit}\PY{p}{(}\PY{n}{X\PYZus{}train}\PY{p}{,} \PY{n}{y\PYZus{}train}\PY{p}{)}
                 \PY{n}{y\PYZus{}pred} \PY{o}{=} \PY{n}{model}\PY{o}{.}\PY{n}{predict}\PY{p}{(}\PY{n}{X\PYZus{}test}\PY{p}{)}
                 \PY{n}{fig}\PY{p}{,} \PY{n}{ax} \PY{o}{=} \PY{n}{plt}\PY{o}{.}\PY{n}{subplots}\PY{p}{(}\PY{n}{figsize}\PY{o}{=}\PY{p}{(}\PY{l+m+mi}{10}\PY{p}{,}\PY{l+m+mi}{5}\PY{p}{)}\PY{p}{)}
                 \PY{n}{fig}\PY{o}{.}\PY{n}{suptitle}\PY{p}{(}\PY{l+s+s2}{\PYZdq{}}\PY{l+s+si}{\PYZob{}0\PYZcb{}}\PY{l+s+se}{\PYZbs{}n}\PY{l+s+s2}{MSE:}\PY{l+s+si}{\PYZob{}1\PYZcb{}}\PY{l+s+s2}{ | EVS:}\PY{l+s+si}{\PYZob{}2\PYZcb{}}\PY{l+s+s2}{\PYZdq{}}\PY{o}{.}\PY{n}{format}\PY{p}{(}\PY{n}{name}\PY{p}{,}\PY{n}{mean\PYZus{}squared\PYZus{}error}\PY{p}{(}\PY{n}{y\PYZus{}test}\PY{p}{,}\PY{n}{y\PYZus{}pred}\PY{p}{)}\PY{p}{,}\PY{n}{explained\PYZus{}variance\PYZus{}score}\PY{p}{(}\PY{n}{y\PYZus{}test}\PY{p}{,}\PY{n}{y\PYZus{}pred}\PY{p}{)}\PY{p}{)}\PY{p}{,} \PY{n}{fontsize}\PY{o}{=}\PY{l+m+mi}{12}\PY{p}{)}
                 \PY{n}{sns}\PY{o}{.}\PY{n}{kdeplot}\PY{p}{(}\PY{n}{y\PYZus{}test}\PY{p}{)}
                 \PY{n}{sns}\PY{o}{.}\PY{n}{kdeplot}\PY{p}{(}\PY{n}{y\PYZus{}pred}\PY{p}{)}
\end{Verbatim}


    \begin{Verbatim}[commandchars=\\\{\}]
{\color{incolor}In [{\color{incolor}44}]:} \PY{k+kn}{from} \PY{n+nn}{xgboost} \PY{k}{import} \PY{n}{XGBRegressor}
         \PY{k+kn}{from} \PY{n+nn}{sklearn}\PY{n+nn}{.}\PY{n+nn}{linear\PYZus{}model} \PY{k}{import} \PY{n}{Lasso}\PY{p}{,}\PY{n}{ElasticNet}\PY{p}{,}\PY{n}{Ridge}
         \PY{k+kn}{from} \PY{n+nn}{sklearn}\PY{n+nn}{.}\PY{n+nn}{svm} \PY{k}{import} \PY{n}{SVR}\PY{p}{,}\PY{n}{LinearSVR}
         \PY{n}{names}\PY{o}{=}\PY{p}{[}\PY{l+s+s2}{\PYZdq{}}\PY{l+s+s2}{XGBoost Linear}\PY{l+s+s2}{\PYZdq{}}\PY{p}{,} \PY{l+s+s2}{\PYZdq{}}\PY{l+s+s2}{XGBoost Tree}\PY{l+s+s2}{\PYZdq{}}\PY{p}{,} \PY{l+s+s2}{\PYZdq{}}\PY{l+s+s2}{Lasso}\PY{l+s+s2}{\PYZdq{}}\PY{p}{,}\PY{l+s+s2}{\PYZdq{}}\PY{l+s+s2}{Elastic Net}\PY{l+s+s2}{\PYZdq{}}\PY{p}{,}\PY{l+s+s2}{\PYZdq{}}\PY{l+s+s2}{Ridge}\PY{l+s+s2}{\PYZdq{}}\PY{p}{]}
         \PY{n}{models} \PY{o}{=} \PY{p}{[}\PY{n}{XGBRegressor}\PY{p}{(}\PY{n}{booster}\PY{o}{=}\PY{l+s+s1}{\PYZsq{}}\PY{l+s+s1}{gblinear}\PY{l+s+s1}{\PYZsq{}}\PY{p}{)}\PY{p}{,}\PY{n}{XGBRegressor}\PY{p}{(}\PY{p}{)}\PY{p}{,}\PY{n}{Lasso}\PY{p}{(}\PY{p}{)}\PY{p}{,}\PY{n}{ElasticNet}\PY{p}{(}\PY{p}{)}\PY{p}{,}\PY{n}{Ridge}\PY{p}{(}\PY{p}{)}\PY{p}{]}
         \PY{n}{train\PYZus{}predict\PYZus{}and\PYZus{}score}\PY{p}{(}\PY{n}{X\PYZus{}train}\PY{p}{,}\PY{n}{y\PYZus{}train}\PY{p}{,}\PY{n}{X\PYZus{}test}\PY{p}{,}\PY{n}{y\PYZus{}test}\PY{p}{,}\PY{n}{models}\PY{p}{,}\PY{n}{names}\PY{p}{)}
\end{Verbatim}


    \begin{Verbatim}[commandchars=\\\{\}]
D:\textbackslash{}Continuum\textbackslash{}anaconda3\textbackslash{}lib\textbackslash{}site-packages\textbackslash{}sklearn\textbackslash{}linear\_model\textbackslash{}coordinate\_descent.py:491: ConvergenceWarning: Objective did not converge. You might want to increase the number of iterations. Fitting data with very small alpha may cause precision problems.
  ConvergenceWarning)
D:\textbackslash{}Continuum\textbackslash{}anaconda3\textbackslash{}lib\textbackslash{}site-packages\textbackslash{}sklearn\textbackslash{}linear\_model\textbackslash{}coordinate\_descent.py:491: ConvergenceWarning: Objective did not converge. You might want to increase the number of iterations. Fitting data with very small alpha may cause precision problems.
  ConvergenceWarning)

    \end{Verbatim}

    \begin{center}
    \adjustimage{max size={0.9\linewidth}{0.9\paperheight}}{output_88_1.png}
    \end{center}
    { \hspace*{\fill} \\}
    
    \begin{center}
    \adjustimage{max size={0.9\linewidth}{0.9\paperheight}}{output_88_2.png}
    \end{center}
    { \hspace*{\fill} \\}
    
    \begin{center}
    \adjustimage{max size={0.9\linewidth}{0.9\paperheight}}{output_88_3.png}
    \end{center}
    { \hspace*{\fill} \\}
    
    \begin{center}
    \adjustimage{max size={0.9\linewidth}{0.9\paperheight}}{output_88_4.png}
    \end{center}
    { \hspace*{\fill} \\}
    
    \begin{center}
    \adjustimage{max size={0.9\linewidth}{0.9\paperheight}}{output_88_5.png}
    \end{center}
    { \hspace*{\fill} \\}
    
    Ez alapján az XGBoost tűnik a leghatékonyabbnak.


    % Add a bibliography block to the postdoc
    
    
    
    \end{document}
